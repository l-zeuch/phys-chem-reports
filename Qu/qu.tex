\section{Theorie}

In diesem Versuch wird der Mechanismus und Ablauf einer nukleophilen Substitution zwischen Brommethan und Chlorid simuliert.
Das Ergebnis dieser Rechnung wird weiterverwendet, um eine Geometrieoptimierung und Frequenzanalyse der Edukte und Produkte durchzuführen.
Es wird nach einem Übergangszustand gesucht, der mittels dem Ergebnis der Frequenzanalyse bestätigt wird.
Die Berechnung wird mit den STO-3G und cc-pVDZ Basissätzen im Hartree-Fock Verfahren durchgeführt.
Die Ergebnisse der zwei Berechnungen werden miteinander verglichen und bewertet.

\subsection{Potentialhyperfläche}

Wird vom Potential des harmonischen Oszillators ausgegangen, so beschreibt eben dieses Potential die Energie zweier Atome $A, B$ für alle Abstände $R_{AB}$ in einer Dimension.
Nun ist intuitiv klar, dass dieses Bild nicht der Realität entspricht, da der Ort eines jeden Atoms über \textit{drei} Raumdimensionen beschrieben wird, also $3N$ Freiheitsgrade.
Da die Atome in einem Molekül miteinander verknüpft sind, macht es keinen Unterschied, ob die Atome nun an $x, y, z$ oder $x + n, y + n, z + n$ anzufinden sind.
Somit reduzieren sich die Freiheitsgrade zu $3N - 3$.

In zweiter Instanz macht die Gesamtdrehung des Moleküls um $x, y, z$ ebenfalls keinen Unterschied für die relative Anordnung der Atome.
Es reduzieren sich die Freiheitsgrade also zu $3N - 6$,  beziehungsweise zu $3N - 5$ für lineare Moleküle, da hier die Drehung um die Molekülachse nicht messbar ist.
Stattdessen können sich lineare Moleküle biegen, um sogar gegebenenfalls eine neue, günstigere Konformation einzunehmen.
Dieser Freiheitsgrad wirkt sich offensichtlich auf die relative Anordnung der Atome aus und wird durch $3N - 5$ berücksichtigt.

Um all diese Freiheitsgrade mathematisch zu erfassen muss das Potential in $3N - 5$ Dimensionen beschrieben werden.
Es wird eine im $3N - 6$-dimensionalem Raum gekrümmte Fläche erhalten.
Diese Fläche wird Potentialhyperfläche genannt.

\subsection{Geometrieoptimierung}

In einer quantenchemischen Rechnung ist es das Ziel, eine stabile, energetisch günstige Molekülstruktur zu finden.
Diese ist durch ein idealerweise globales Minimum auf der Potentialhyperfläche beschrieben, allerdings findet sich häufiger ein lokales Minimum.
In einer Geometrieoptimierung werden hierfür die partiellen Ableitungen der Hyperfläche nach den Koordinaten berechnet.
Anhand diesen Ableitungen kann erkannt werden, wie einzelne $R$ verändert werden müssen, um einer stabilen Struktur näher zu kommen.

\subsection{Frequenzanalyse}

Werden zusätzlich die zweiten Ableitungen des Potentials $V$ nach den  Koordinaten $R$ berechnet --- die ohnehin notwendig sind um Minima eindeutig zu bestimmen --- so können die Eigenschwingungen $\nu$ des Moleküls identifiziert werden.
Da in unmittelbarer Nähe des Minimums das Hooke'sche Gesetz gilt, ist der Zusammenhang in einer Dimension recht trivial. 
In mehreren Dimensionen (also einer \enquote{echten} Berechnung) handelt es sich um ein Eigenwertproblem, was quantenchemische Programme lösen können.
\begin{align}
    \nu = \sqrt{k\mu} \,,\, k = \odv[order=2]{V}{R}
\end{align}

Sind \textit{alle} zweiten Ableitungen positiv, wurde also ein Minimum gefunden, so sind auch die resultierenden Frequenzen reell.
Ist dagegen genau eine zweite Ableitung negativ, entspricht der gefundene Punkt einem Übergangszustand; die Frequenz wird imaginär.
Hieraus lassen sich sogar gezielt Übergangszustände des Moleküls finden, indem zunächst solch eine imaginäre Eigenschwingung identifiziert wird und alle anderen Koordinaten daraufhin optimiert werden.
Diese Verfahren werden als Frequenzanalyse bezeichnet.

\subsection{Born-Oppenheimer Näherung}

Quantenchemische Berechnungen basieren auf der Schrödinger-Gleichung (SG).
Diese ist für Systeme mit mehreren Elektronen und Atomkernen nicht mehr analytisch zu lösen, muss also iterativ genähert werden.
Aus Sicht der Elektronen erscheinen Atomkerne an festen Orten, da erstere um ein Vielfaches schneller und leichter sind,
wodurch für eine Lösung der SG die Atomkerne als fixe Punkte betrachtet werden können, was die Berechnungen vereinfacht.

\subsection{Hartree-Fock}

Die Wellenfunktionen der Elektronen in einem Molekül folgen aus der Lösung der Schrödinger-Gleichung des Systems.
Da diese --- wie oben beschrieben --- für Mehrelektronensysteme nur numerisch lösbar ist, muss genähert werden.
Hierzu wird die \textit{Orbitalnäherung} verwendet, womit die Gesamtwellenfunktion als Produkt der einzelnen Wellenfunktionen beschrieben wird.
Damit die Wellenfunktion antisymmetrisch bei Vertauschung zweier Elektronen ist, wird diese durch eine \textit{Slater-Determinante} ausgedrückt.

Das Hartree-Fock-Verfahren basiert nun auf der Beschreibung solch einer Wellenfunktion in Kombination mit dem Variationstheorem, welches besagt, dass der Energie-Erwartungswert einer genäherten Wellenfunktion stets größer oder gleich der tatsächlichen Energie ist.
Als Variationsansatz in Hartree-Fock werden nun die einzelnen Orbitale so lange verändert, bis ein Minimum gefunden wird.

Die einzelnen Rechenschritte liefern Eigenwertprobleme, die es zu lösen gilt.
Die Lösung liefert neue MO-Koeffizienten, die nun für die nächste Iteration verwendet werden, so lange, bis sich die Energie nicht weiter verändert.
Quantenchemische Programme führen diese iterative Rechnung nach gegebenen Parametern aus.

\section{Durchführung}

Zur Berechnung wurde das Quantenchemie-Programm \textit{Gaussian 16}\autocite{g16} verwendet, gekoppelt mit \textit{GaussView}\autocite{gv6} zur Bedienung.
Zu Beginn des Versuches wurde die C-Cl-Bindungslänge als Scankoordinate gewählt, im Intervall von \qty{1.5}{\angstrom} und \qty{3.0}{\angstrom} mit einer Schrittweite von \qty{0.1}{\angstrom}.
Für die Berechnung wurde zunächst der STO-3G Basissatz verwendet.
Aus dem Ergebnis der Scanrechnung wurde in \textit{GaussView} der Zustand angewählt, der die niedrigste Energie aufwies.
Ausgehend von diesem lokalen Minimum wurde eine Geometrieoptimierung und Frequenzanalyse durchgeführt.
Das Ergebnis der Frequenzanalyse wurde weiterverwendet, um eine abschließende Übergangszustandsuche durchzuführen.
Hierzu wurde der Zustand angewählt, der die höchste Energiedifferenz zwischen Edukten und Produkten und genau eine imaginäre Eigenfrequenz aufwies.

Der Versuch wurde mit dem cc-pVDZ Basissatz wiederholt.

\newpage

\section{Ergebnisse und Diskussion}

\subsection{Scanrechnung}

Die Scanrechnungen ergaben die in \autoref{plot-energs} dargestellten Verläufe.
Es ist gut zu erkennen, dass der größere cc-pVDZ Basisatz die Gesamtenergie der Produkte deutlich niedriger als die Edukte berechnet,
wobei der STO-3G Basisatz die Gesamtenergie der Produkte sogar über die Gesamtenergie der Edukte berechnet hat.
Ferner ist die Energiedifferenz zwischen Übergangszustand und Produkten im cc-pVDZ ausgeprägter als im STO-3G.
Insgesamt sind alle Energien im cc-pVDZ niedriger als bei STO-3G.

Diese Unterschiede zeigen deutlich die größere Präzision des cc-pVDZ Basissatzes gegenüber STO-3G.
Während STO-3G nur drei Gauß-Orbitale zur Annäherung verwendet, verwendet cc-pVDZ zusätzlich Polarisationsfunktionen\autocite{dunning-basis-set},
was aufgrund der verbesserten angularen Flexibilität die Absenkung der Orbitalenergien unterstützt.\autocite{thc2}

\begin{figure}[H]
    \centering
    \begin{subfigure}{0.95\textwidth}
        \includegraphics[width=\textwidth]{Qu/Plot_Energie_STO-3G.png}
        \caption{STO-3G}
        \label{plot-sto3g}
    \end{subfigure}
    \begin{subfigure}{0.95\textwidth}
        \includegraphics[width=\textwidth]{Qu/Plot_Energie_cc-pVDZ.png}
        \caption{cc-pVDZ}
        \label{plot-ccpVDZ}
    \end{subfigure}
    \caption{Auftragungen der berechneten Gesamtenergie entlang der Bindungslänge im Intervall $I$ = (\qty{1.5}{\angstrom}, \qty{3.0}{\angstrom}).
        Schrittweite \qty{0.1}{\angstrom}.
        Verwendete Basissätze STO-3G (\autoref{plot-sto3g}) und cc-pVDZ (\autoref{plot-ccpVDZ}).
    }
    \label{plot-energs}
\end{figure}

\subsection{Reaktionsenthalpie}

Die Reaktionsenthalpie $\Delta_r H^0$ berechnet sich nach \cite{gaussian-thermochem} aus der Differenz der \textit{Sum of electronic and thermal Enthalpies} der Produkte und Edukte.
\begin{align*}
              \Delta_r H^0(298.15\, \text{\unit{\kelvin}}) &= \sum (\mathcal{E}_0 + H_{corr})_{\text{Produkte}} - \sum (\mathcal{E}_0 + H_{corr})_{\text{Edukte}} \\
     \Delta_r H^0_{\text{STO-3G}} &= (- 3038.2206 - (- 3038.2556))\, \text{E\textsubscript{h}} \\
                           &=  0.0350\, \text{E\textsubscript{h}} \\
                           &= 91.89\, \text{\unit{\kilo\joule\per\mole}} \\
    \Delta_r H^0_{\text{cc-pVDZ}} &= (- 3071.5283 - (- 3071.5161))\, \text{E\textsubscript{h}} \\
                           &= - 0.0077\, \text{E\textsubscript{h}} \\
                           &= - 20.23\, \text{\unit{\kilo\joule\per\mole}}
\end{align*}
Anhand der Literaturwerte $\Delta H_f^0$ für Chlormethan (\qty{- 81.9}{\kilo\joule\per\mole})\autocite{crc:std-thermo:ch3cl+ch3br}, Brommethan (\qty{- 35.4}{\kilo\joule\per\mole})\autocite{crc:std-thermo:ch3cl+ch3br},
Chlorid (\qty{- 233.9}{\kilo\joule\per\mole})\autocite{nist-thermochem}, und Bromid (\qty{-219.0}{\kilo\joule\per\mole})\autocite{nist-thermochem}
berechnet sich die Reaktionsenthalpie zu \qty{- 30.7}{\kilo\joule\per\mole}.

\subsection{Reaktionsentropie}

Die Reaktionsentropie $\Delta S_0^R$ berechnet sich aus der Summe der einzelnen Gesamtentropien $S_{0i}$ multipliziert mit dem entsprechenden stöchiometrischen Faktor $\nu_i$.
\begin{align*}
              \Delta S_0^R &= \sum_i \nu_i S_{0i} \\
    \Delta S_{0, \text{STO-3G}}^R &= (70.293 - 68.037) \, \text{\unit{\cal\per\mole\per\kelvin}} \\
                           &= 2.256 \, \text{\unit{\cal\per\mole\kelvin}} = 9.4391 \, \text{\unit{\joule\per\mole\per\kelvin}} \\
   \Delta S_{0, \text{cc-pVDZ}}^R &= (79.057 - 76.443) \, \text{\unit{\cal\per\mole\per\kelvin}} \\
                           &= 2.614 \, \text{\unit{\cal\per\mole\per\kelvin}} = 10.937 \, \text{\unit{\joule\per\mole\per\kelvin}}
\end{align*}
Anhand der Literaturwerte $S_0$ für Chlormethan (\qty{234.6}{\joule\per\mole\per\kelvin})\autocite{crc:std-thermo:ch3cl+ch3br}, Brommethan (\qty{246.4}{\joule\per\mole\per\kelvin})\autocite{crc:std-thermo:ch3cl+ch3br},
Chlorid (\qty{153.4}{\joule\per\mole\per\kelvin})\autocite{nist-thermochem}, und Bromid (\qty{163.5}{\joule\per\mole\per\kelvin})\autocite{nist-thermochem}
berechnet sich die Reaktionsentropie zu \qty{-1.7}{\joule\per\mole\per\kelvin}.

\subsection{Freie Reaktionsenthalpie}

Analog berechnet sich die freie Reaktionsenthalpie $\Delta_r G^0$ aus der Differenz der \textit{Sum of electronic and thermal Free Energies} der Produkte und Edukte.
\begin{align*}
    \Delta_r G^0(298.15\, \text{\unit{\kelvin}}) &= \sum (\mathcal{E}_0 + G_{corr})_{\text{Produkte}} - \sum (\mathcal{E}_0 + G_{corr})_{\text{Edukte}} \\
     \Delta_r G^0_{\text{STO-3G}} &= (- 3038.2578 - (- 3038.2879))\, \text{E\textsubscript{h}} \\
                           &=  0.0301\, \text{E\textsubscript{h}} \\
                           &= 79.03\, \text{\unit{\kilo\joule\per\mole}} \\
    \Delta_r G^0_{\text{cc-pVDZ}} &= (- 3071.5659 - (- 3071.5525))\, \text{E\textsubscript{h}} \\
                           &= - 0.0134\, \text{E\textsubscript{h}} \\
                           &= - 35.18\, \text{\unit{\kilo\joule\per\mole}}
\end{align*}
Über die Gibbs-Helmholtz-Gleichung wird die freie Reaktionsenthalpie folgendermaßen berechnet.
\begin{align*}
    \Delta G &= \Delta H - T \cdot \Delta S \\
             &= -30.7\, \text{\unit{\kilo\joule\per\mole}} - 298.15\, \text{\unit{\kelvin}} \cdot (-1.7\, \text{\unit{\joule\per\mole\per\kelvin}}) \\
             &= -30.65\, \text{\unit{\kilo\joule\per\mole}}
\end{align*}

Die Überschätzungen der Enthalpien und der Entropie sind insbesondere bei STO-3G auf die recht ungenaue Näherung der Orbitale zurückzuführen.
Der cc-pVDZ Basisatz weist eine deutlich kleinere Abweichung auf; diese Abweichung ist höchstwahrscheinlich darauf zurückzuführen, dass für Atome der dritten Periode und darüber hinaus
größere Basissätze (zum Beispiel cc-pV(D+d)Z) notwendig sind.\autocite{al-ar:large-basis}

Für noch größere Atome kann es unter Umständen sogar nötig sein, Pseudopotential-Basissätze (cc-pVDZ-PP) oder auch re\-la\-ti\-vis\-tisch-kontrahierte Douglas-Kroll-Basissätze (cc-pVDZ-DK) zu verwenden.\autocite{ga-kr:bigger-basis}
Ochterski suggeriert zudem in seiner Anleitung zur Thermochemie in \textit{Gaussian}, dass für alle einzelnen Edukte und Produkte die Geometrieoptimierung durchzuführen sei\autocite{gaussian-thermochem},
was in diesem Versuch nicht der Fall war.

\subsection{Gleichgewichtskonstante}

Die Gleichgewichtskonstante $K$ berechnet sich aus $\Delta_r G^0$ wie folgt. $R$ ist die universelle Gaskonstante.
\begin{align*}
                    K &= \exp\left(- \frac{\Delta_r G^0}{RT} \right) \\
    K_{\text{STO-3G}} &= \exp\left(- \frac{79.3\, \text{\unit{\kilo\joule\per\mole}}}{R \cdot 298.15\, \text{\unit{\kelvin}}} \right) \\
                      &= 1.032 \\
   K_{\text{cc-pVDZ}} &= \exp\left(- \frac{-35.18\, \text{\unit{\kilo\joule\per\mole}}}{R \cdot 298.15\, \text{\unit{\kelvin}}} \right) \\
                      &= 0.985
\end{align*}

\subsection{Übergangszustand}

Die Symmetriegruppe des Übergangszustandes wurde bestimmt zu $C_1$.
Dies entspricht nicht der Erwartung von $C_3$, da im Übergangszustand der Tetraeder umschwingt und durch die drei Wasserstoff-Atome eine dreizählige Drehachse mit Cl-C-Br auf dieser Achse darstellt.

\begin{align*}
    \Delta G^{\ddagger} &= \sum (\mathcal{E}_0 + H_{corr})_{\text{TS}} - \sum (\mathcal{E}_0 + H_{corr})_{\text{Edukte}} \\
    \Delta G^{\ddagger}_{\text{STO-3G}} &= (- 3028.2434 - (- 3038.2879))\, \text{E\textsubscript{h}} \\
                                        &= 0.0445\, \text{E\textsubscript{h}} = 116.83\, \text{\unit{\kilo\joule\per\mole}} \\
   \Delta G^{\ddagger}_{\text{cc-pVDZ}} &= (- 3071.5391 - (- 3071.5524))\, \text{E\textsubscript{h}} \\
                                        &= 0.0133\, \text{\textsubscript{h}} = 34.92\, \text{\unit{\kilo\joule\per\mole}} \\
                               \Delta S_0^R &= \sum_i \nu_i S_{0i} \\
    \Delta S_{0,\text{STO-3G}}^{R,\ddagger} &= (70.353 - 78.215) \text{\unit{\cal\per\mole\per\kelvin}} \\
                                            &= - 7.862\, \text{\unit{\cal\per\mole\per\kelvin}} = - 32.895\, \text{\unit{\joule\per\mole\per\kelvin}} \\
    \Delta S_{0,\text{STO-3G}}^{R,\ddagger} &= (73.322 - 79.057) \text{\unit{\cal\per\mole\per\kelvin}} \\
                                            &= - 5.737\, \text{\unit{\cal\per\mole\per\kelvin}} = - 24.003\, \text{\unit{\joule\per\mole\per\kelvin}} \\          
\end{align*}

\newpage

\section{Zusammenfassung}

Die nukleophile Substitution zwischen Brommethan und Chlorid wurde simuliert.
Hierzu wurde die Potentialhyperfläche entlang der C-Cl-Bindungslänge berechnet.
Anhand der Scanrechnung wurden Geometrieoptimierungen und Frequenzanalysen der Edukte, Produkte und Übergangszustände durchgeführt.
Die Berechnungen wurden mit den Basissätzen STO-3G und cc-pVDZ im Hartree-Fock Verfahren durchgeführt.
Die Ergebnisse der zwei Berechnungen wurden miteinander verglichen und bewertet.
Die Reaktionsenthalpie, Reaktionsentropie, freie Reaktionsenthalpie und Gleichgewichtskonstante wurden berechnet und mit Literaturwerten verglichen.
Die Ergebnisse zeigten, dass der cc-pVDZ Basissatz deutlich präzisere Ergebnisse lieferte als STO-3G.

\newpage
\printbibliography
