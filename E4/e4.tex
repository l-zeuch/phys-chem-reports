\section{Einleitung}
Die Elektrochemie ist ein grundlegender Bereich der Physikalischen Chemie, der den Zusammenhang zwischen elektrischer Energie und chemischen Reaktionen untersucht.
Sie bildet die Grundlage moderner Technologien und ist ein wichtiges Werkzeug zur Analyse von Redoxprozessen.
Unter den elektrochemischen Methoden ist die Cyclovoltammetrie (CV) besonders vielseitig, um Elektronentransferreaktionen und Stofftransportvorgänge zu untersuchen.

Die CV ist ein Messverfahren, bei dem das Potential einer Elektrode linear gesteigert wird.
Nachdem das Potential ein gesetztes Maximum erreicht hat, wird dieses nun wieder bis zum Ausgangspotential herabgesetzt, es wird also eine Dreieckspannung erzeugt.
Dabei wird eine Drei-Elektroden-Zelle verwendet: Referenz-, Reaktions-, und Gegenelektrode.
Zwischen Reaktions- und Referenzelektrode wird das Potential gemessen, zwischen Reaktions- und Gegenelektrode der Strom.
Es wird der Strom gegen das Potential aufgetragen und man erhält einen zyklischen Verlauf.
Die Messung wird in der Regel so lange wiederholt, bis die voltammetrische Kurve einen zyklischen Gleichgewichtszustand erreicht.

Dieses Verfahren wird besonders zur Untersuchung von Redoxsystemen verwendet, um beispielsweise die Anwesenheit von Zwischenstufen oder auch die Umkehrbarkeit einer Reaktion zu bestimmen.
In der Lebensmittelchemie findet die Cyclovoltammetrie Anwendung zur Bestimmung der antioxidativen Kapazität (Oxygen Radical Absorbance Capacity, \textit{ORAC}) von Nahrung,\autocite{chevion2000antioxidant}$^, $\autocite{brcanovic2013cyclic}
da insbesondere niedermolekulare Antioxidanzen von besonderer Bedeutung sind, weil diese vermutlich das Risiko für Krebs\autocite{sohal1996oxidative}, Schlaganfälle\autocite{leinonen2000low},
und neurodegenerative Krankheiten\autocite{gilbert2000fifty} senken.

In diesem Versuch wird das elektrochemische Verhalten von Ferrocen in \qty{0.1}{\mol\per\liter} \ch{TBAF}/Acetonitril mittels CV untersucht.
Es werden wichtige Redox-Parameter bestimmt, die Abhängigkeit des Stroms von der Scanrate untersucht, und die Reversibilität der Reaktion geprüft sowie mit theoretischen Erwartungen verglichen.
Ferner wird der Diffusionskoeffizent von Ferrocen berechnet und mit der Literatur abgeglichen, um die Qualität der Messungen zu bewerten.

Das System Ferrocen/Ferrocenium (\ch{Fc}/\ch{Fc+}) ist ein gut untersuchtes Ein-Elektronen-Redoxpaar und dient in der Elektrochemie häufig als Referenzsystem\autocite{elgrishi2018practical}.
Daher eignet es sich besonders gut für einen Einführungsversuch in die Cyclovoltammetrie.

\newpage
\section{Theorie}
Das Elektrodenpotential wird dreieckförmig verändert, der Stromverlauf zeigt die entsprechenden Oxidations- und Reduktionsprozesse.

Während des Vorwärtsscans wird das Potential positiver, ab einem bestimmten Punkt wird das Ferrocen an der Reaktionselektrode nach \autoref{sch:ferrocen_redox} zu Ferrocenium oxidiert.
Der Strom steigt weiter, bis durch Diffusion die Konzentration der Reaktionsteilnehmer an der Elektrodenoberfläche begrenzt wird.
Dies führt zum anodischen Spitzenstrom $i_{pa}$.
Nach Umkehr des Scans wird das Potential negativer, das zuvor erzeugte Ferrocenium wird wieder zu Ferrocen reduziert, es wird der kathodische Spitzenstrom $i_{pc}$ erhalten.
\begin{scheme}[H]
    \centering
    \schemestart
    \ch{Fe(C5H5)2} \ch{ <=> } \ch{Fe(C5H5)2^{+}} + \ch{e^{-}}
    \schemestop
    \caption{Redoxreaktion von Ferrocen (\ch{Fc}) zu Ferocenium (\ch{Fc+}).}
    \label{sch:ferrocen_redox}
\end{scheme}

Nach Randles-\v{S}ev\v{c}ik ist der maximale Strom $i_p$ für ein reversiblen, diffusionsbegrenzten Prozess bei $T= \qty{25}{\degreeCelsius}$ gegeben durch
\begin{equation}
    i_p = (\num{2.69e5}) n^{3/2} A D^{1/2} C v^{1/2} \label{eq:randles_sevcik}
\end{equation}
wobei $n$ die Anzahl der übertragenen Elektronen, $A$ die Elektrodenoberfläche in \unit{\centi\meter\squared}, $D$ der Diffusionskoeffizient in \unit{\centi\meter\squared\per\second},
$C$ die Konzentration in \unit{\mole\per\centi\meter\cubed} und $v$ die Scanrate in \unit{\volt\per\second} ist.

Durch Auftragung von $i_{pa}$ gegen $v^{1/2}$ kann der Diffusionskoeffizient $D$ mittels einer linearer Regression bestimmt werden.

Für eine reversible Reaktion bei Ein-Elektronen-Systemen ($n = 1$) gilt, dass die Differenz zwischen Oxidations- und Reduktionspotential $\Delta E_p$ bei Raumtemperatur \qty{59}{\milli\volt} beträgt.
Dieser Zusammenhang ergibt sich aus der vereinfachten Nernst-Gleichung bei $T = \qty{298.15}{\kelvin}$, und der Tatsache, dass bei reversiblen Reaktionen die Konzentrationen der oxidierten und reduzierten Spezies am Scheitelpunkt des Stroms gleich sind.
\begin{align}
    \Delta E &= E^0 + \frac{\qty{59}{\milli\volt}}{n} \cdot \underbrace{\log \mleft( \frac{c_{\text{ox}}}{c_{\text{red}}} \mright)}_{\text{= 1}} \\
\intertext{Ferner ist das Verhältnis der Stromstarken ungefähr gleich.}
        1 &\approx \frac{i_{pa}}{i_{pc}} \\
\intertext{Zusätzlich gilt für das formelle Potential}
        E^{0'} &= \frac{E_{pa} + E_{pc}}{2}
\end{align}
Während der Reaktion wandern die gelösten Ionen ständig zwischen den Elektroden.
Um diese Ionenwanderung zu unterstützen, wird der Lösung ein Salz, das Stützelektrolyt, beigefügt.
In diesem Versuch ist dieses Stützelektrolyt Tetra-n-butylammoniumfluorid (\ch{TBAF}), ein quartäres Ammoniumsalz.
\begin{figure}[H]
    \centering
    \chemfig[angle increment=30, atom sep=1.25em]{
        \ch{N+}(% right
            -[-1]-[-3]-[-1]-[-3]
        )
        ( % bottom
            -[-4]-[-6]-[-4]-[-6]
        )
        ( % left
            -[-7]-[-9]-[-7]-[-9]
        )
        (%top
            -[2]-[0]-[2]-[0]
        )
    }
    \chemfig{\ch{F+}}
    \caption{Struktur von Tetra-n-butylammoniumfluorid (TBAF).}
\end{figure}
Ein reversibles System wird dadurch ausgezeichnet, dass die untersuchte Substanz nach der Reduktion stabil ist, also anschließend wieder rückoxidiert werden kann.
Spezifisch angewandt auf die Elektrochemie bedeutet das, dass der Elektronentransfer eine geringe Barriere aufweist, sich also schnell ein Gleichgewichtszustand einstellen kann.
Dies zeichnet sich durch einen kleinen Unterschied zwischen den Oxidations- und Reduktionspotentialen ($\Delta E_p$) aus, sowie durch ähnliche Stromstärken für Oxidation und Reduktion.

Wasser als Löungsmittel ist für die Cyclovoltammetrie nur bedingt geeignet, da es zur Elektrolyse kommen kann, was die Messung beeinflusst.
Insbesondere bei höheren Potentialen kann es zur Sauerstoffentwicklung an der Anode kommen und somit zu unerwünschten Oxidationsprozessen, gegebenenfalls sogar zur Zerstörung der Elektrode beziehungsweise der Probe.

\newpage

Zur Auswertung der Daten, insbesondere zur Bestimmung der Spitzenströme $i_{pa}$ und $i_{pc}$, ist es erforderlich, den kapazitiven Anteil des Stroms bei der Umkehr des Potentials zu berücksichtigen.
Dies wird durch eine angenäherte lineare Basislinie im Bereich nach den Umkehrpunkten des Potentials erreicht, die dann vom gemessenen Strom subtrahiert wird.
Diese Basislinien sind in \autoref{fig:cv_example} (sowie allen weiteren Voltammogrammen) als rote beziehungsweise graue Linie dargestellt.

Dieser Zusammenhang ist dadurch zu erklären, dass sich an der Elektrodenoberfläche Elektronen ansammeln, die bei Umkehr des Potentials zunächst entfernt werden müssen, bevor die eigentliche Redoxreaktion stattfinden kann.
Im Grunde agiert die Elektrode also wie ein Kondensator, der während des Scans aufgeladen wird und bei Potentialumkehr entladen wird.

\begin{figure}[H]
    \centering
    \includegraphics[width=\textwidth]{E4/plots/100mVs.png}
    \caption{Beispiel eines Cyclovoltammogramms (\qty{100}{\milli\volt\per\second}) mit Markierung der relevanten Parameter.}
    \label{fig:cv_example}
\end{figure}

\newpage
\section{Durchführung}
\subsection{Sicherheitshinweise}
\begin{itemize}
    \item Acetonitril ist giftig und leicht entzündlich: Haut- und Augenkontakt vermeiden. 
    \item Lösungen im vorgesehenen Abfallbehälter entsorgen.
    \item Ferrocen-Lösung ist wenig gefährlich, sollte aber vorsichtig behandelt werden.
    \item Mit Standard-Schutzausrüstung arbeiten (Laborkittel, Handschuhe, Schutzbrille).
\end{itemize}

Eine Glaskohlenstoff-Elektrode ($d = \qty{3.0}{\milli\meter}, A = \qty{0.071}{\cm\squared}$) wurde poliert, gespült, und getrocknet.
Es wurde eine \ch{Ag}/\ch{Ag+}-Referenzelektrode verwendet.

Es wurden \qty{15}{\milli\liter} einer \qty{1}{\milli\mol\per\liter} Ferrocen-Lösung in \qty{0.1}{\mol\per\liter} TBAF / Acetonitril in die Drei-Elektroden-Zelle gefüllt.
Der Potentionstat zur Erzeugung der Dreieckspannung wurde auf einen Potentialbereich von \qty{-0.2}{\volt} bis \qty{0.8}{volt}, ein Startpotential von \qty{0.0}{\volt},
und einer Scanrate von \qty{10}{\milli\volt\per\second} eingestellt.
Die Messung wurde gestartet und aufgezeichnet.

Die Messung wurde mit Scanraten von 25, 50, 75, 100, 150, 200, 300, 400, und \qty{500}{\milli\volt\per\second} wiederholt.
Zusätzlich wurde eine Messung bei \qty{100}{\milli\volt\per\second} aufgenommen, während die Lösung gerührt wurde.

\section{Ergebnisse und Diskussion}
Zur Auswertung und Darstellung der Daten wurde Python 3.14 mit den externen Bibliotheken NumPy\autocite{numpy} und Matplotlib\autocite{matplotlib} verwendet.

Zunächst veranschaulicht \autoref{fig:100mVs_comparison} den Einfluss des Rührens auf das Cyclovoltammogramm bei einer Scanrate von \qty{100}{\milli\volt\per\second}.
Es ist deutlich zu erkennen, dass das Rühren einen erheblichen Einfluss auf den Verlauf des Cyclovoltammogramms hat, insbesondere im Bereich des maximalen Potentials.
Um ein gutes Cyclovoltammogramm zu erhalten, ist es wichtig, durch die Diffusion der Reaktenden begrenzt zu sein, was durch Rühren gestört wird.
Ferner wird durch den ungenauen Potentialverlauf bei gerührter Lösung die Bestimmung der Redox-Parameter erschwert, zu erkennen an den größeren Unsicherheiten der kapazitiven Basislinie.

In der weiteren Auswertung werden daher nur die Messungen ohne Rühren berücksichtigt.
\begin{figure}[H]
    \centering
    \begin{subfigure}{\textwidth}
        \centering
        \includegraphics[width=\textwidth]{E4/plots/100mVs_stirred.png}
        \caption{Gerührte Lösung.}
    \end{subfigure}
    \begin{subfigure}{\textwidth}
        \centering
        \includegraphics[width=\textwidth]{E4/plots/100mVs.png}
        \caption{Ungerührte Lösung.}
    \end{subfigure}
    \caption{Vergleich Cyclovoltammogramme bei \qty{100}{\milli\volt\per\second} mit und ohne Rühren.}
    \label{fig:100mVs_comparison}
\end{figure}

\begin{table}[H]
    \begin{adjustwidth}{-2.2cm}{-1cm}
    \setlength{\tabcolsep}{4pt}
    \centering
    \caption{Gemessene und berechnete Redox-Parameter von Ferrocen bei verschiedenen Scanraten.}
    \begin{tabular}{c c c c c c c c}
        \toprule
        Scanrate & $E_{pa}$ & $E_{pc}$ & $\Delta E_p$ & $E^{0'}$ & $i_{pa}$ & $i_{pc}$ & $\frac{i_{pa}}{i_{pc}}$ \\ \midrule
        \unit{\milli\volt\per\second} & \unit{\milli\volt} & \unit{\milli\volt} & \unit{\milli\volt} & \unit{\milli\volt} & \unit{\micro\ampere} & \unit{\micro\ampere} & \\ \midrule
        10 & \num{464.5 +- 2.4} & \num{393.9 +- 1.5} & \num{70.6 +- 3.9} & \num{429.3 +- 13.5} & \num{10.27 +- 0.12} & \num{10.29 +- 0.8} & \num{0.998 +- 0.08} \\ \midrule
        25 & \num{468.4 +- 0.01} & \num{401.4 +- 0.01} & \num{66.93 +- 0.01} & \num{434.9 +- 12.7} & \num{15.05 +- 0.09} & \num{14.1 +- 0.4} & \num{1.06 +- 0.03} \\ \midrule
        50 & \num{468.41 +- 0.01} & \num{397.6 +- 1.3} & \num{70.8 +- 1.3} & \num{433.0 +- 13.4} & \num{20.6 +- 0.1} & \num{19.5 +- 0.1} & \num{1.053 +- 0.007} \\ \midrule
        75 & \num{473.34 +- 0.01} & \num{396.39 +- 0.01} & \num{76.95 +- 0.02} & \num{434.9 +- 14.6} & \num{24.6 +- 0.3} & \num{23.7 +- 0.2} & \num{1.04 +- 0.02} \\ \midrule
        100 & \num{463.49 +- 0.01} & \num{391.51 +- 0.02} & \num{71.98 +- 0.04} & \num{427.5 +- 13.6} & \num{28.8 +- 0.5} & \num{27.8 +- 0.2} & \num{1.04 +- 0.02} \\ \midrule
        150 & \num{463.48 +- 0.01} & \num{388.9 +- 1.5} & \num{74.5 +- 1.5} & \num{426.2 +- 14.1} & \num{34.8 +- 0.7} & \num{33.6 +- 0.3} & \num{1.04 +- 0.02} \\ \midrule
        200 & \num{463.55 +- 0.02} & \num{386.34 +- 0.03} & \num{77.21 +- 0.04} & \num{424.9 +- 14.6} & \num{39.8 +- 0.7} & \num{38.6 +- 0.3} & \num{1.03 +- 0.02} \\ \midrule
        300 & \num{464.8 +- 1.2} & \num{383.8 +- 1.5} & \num{81.0 +- 2.7} & \num{424.3 +- 15.4} & \num{48.3 +- 0.8} & \num{46.9 +- 0.3} & \num{1.03 +- 0.02} \\ \midrule
        400 & \num{468.39 +- 0.02} & \num{381.47 +- 0.02} & \num{86.93 +- 0.04} & \num{424.9 +- 16.5} & \num{55.4 +- 0.9} & \num{54.2 +- 0.4} & \num{1.02 +- 0.02} \\ \midrule
        500 & \num{468.42 +- 0.01} & \num{377.7 +- 1.3} & \num{90.7 +- 1.3} & \num{423.1 +- 17.2} & \num{62.0 +- 1.2} & \num{60.9 +- 0.4} & \num{1.02 +- 0.02} \\
        \bottomrule
    \end{tabular}
    \label{tab:redox_parameters}
    \setlength{\tabcolsep}{6pt}
    \end{adjustwidth}
\end{table}
\subsection{Redox-Parameter und Reversibilität}

\autoref{tab:redox_parameters} fasst die gemessenen und berechneten Redox-Parameter von Ferrocen bei den verschiedenen Scanraten zusammen.
Es ist zu erkennen, dass die Verhältnisse der Spitzenströme $i_{pa}$ und $i_{pc}$ nahe bei 1 liegen, was auf eine reversible Reaktion hinweist.
Allerdings weicht die Differenz der Potentiale $\Delta E_p$ deutlich von dem erwarteten Wert von \qty{59}{\milli\volt} ab, insbesondere bei höheren Scanraten.
Dies könnte auf kinetische Einschränkungen oder ohmsche Verluste im System hinweisen, die bei höheren Scanraten stärker ins Gewicht fallen, weil die Reaktionszeit kürzer ist.
Aus der Auftragung der $\Delta E_p$-Werte gegen die Scanrate (siehe Anhang, oben) verdetlicht sich dieser Trend.

Ferner wurde die Temperatur der Lösung nicht kontrolliert, was ebenfalls zu Abweichungen führen kann.
So ist davon auszugehen, dass die Lösung nicht bei \qty{25}{\degreeCelsius} vorlag, was die Nernst-Gleichung beeinflusst.

Da die Verhältnisse der Spitzenströme $i_{pa}$ und $i_{pc}$ jedoch nahe bei 1 liegen, und die Form der Cyclovoltammogramme typisch für reversible Reaktionen ist,
kann insgesamt von einer reversiblen Redoxreaktion ausgegangen werden.

\subsection{Diffusionskoeffizient}
Zur Bestimmung des Diffusionskoeffizienten $D$ wurde der anodische Spitzenstrom $i_{pa}$ gegen die Quadratwurzel der Scanrate $v^{1/2}$ aufgetragen (siehe Anhang, unten).
Die lineare Regression liefert eine Steigung von $m = \num{2.779 +- 0.03} \, \unit{\micro\ampere\per(\milli\volt\per\second)^{1/2}}$.
Nach Umstellung der Randles-\v{S}ev\v{c}ik-Gleichung (\autoref{eq:randles_sevcik}) ergibt sich für den Diffusionskoeffizienten $D$,
mit einer Elektrodenfläche $A = \qty{0.071}{\cm\squared}$ und Konzentration $C = \qty{1e-4}{\mol\per\cm\cubed}$ für einen Ein-Elektronen-Prozess ($n = 1$):
\begin{align*}
    i_p &= (\num{2.69e5}) n^{3/2} A D^{1/2} C v^{1/2} \tag{\ref{eq:randles_sevcik} Wdh.} \\
    D^{1/2} &= \frac{i_p}{(\qty{2.69e5}{\coulomb\per\mol\per(\volt)^{1/2}}) A C} \\
    D &= \mleft( \frac{i_p}{(\qty{2.69e5}{\coulomb\per\mol\per(\volt)^{1/2}}) A C} \mright)^2 \\
      &= \mleft( \frac{m}{(\qty{2.69e5}{\coulomb\per\mol\per(\volt)^{1/2}}) \cdot \qty{0.071}{\cm\squared} \cdot \qty{1e-4}{\mol\per\cm\cubed}} \mright)^2 \\
      &= \mleft( \frac{\qty{2.779 +- 0.03e-6}{\ampere\per(\milli\volt\per\second)^{1/2}}}{\qty{2.69e5}{\coulomb\per\mol\per(\volt)^{1/2}} \cdot \qty{0.071}{\cm\squared} \cdot \qty{1e-4}{\mol\per\cm\cubed}} \mright)^2 \\
    D &= \qty{2.14 +- 0.05e-5}{\cm\squared\per\second}
\end{align*}

Dieser Wert stimmt sehr gut mit dem Literaturwert von $D_{lit} = \qty{2.12e-5}{\cm\squared\per\second}$\autocite{wang2010measurement} überein, was auf eine hohe Qualität der Messungen hinweist.

\subsection{Einfluss der Scanrate}
Die Abbildungen \ref{fig:10mVs} bis \ref{fig:500mVs} zeigen Cyclovoltammogramme bei den verschiedenen Scanraten.
Es ist zu beobachten, dass mit steigender Scanrate die Peaks enger zusammenrücken und die Spitzenströme zunehmen.
Dies ist darauf zurückzuführen, dass bei höheren Scanraten die Zeit für die Diffusion der Reaktanden zur Elektrode kürzer ist, was zu höheren Konzentrationsgradienten und somit zu höheren Strömen führt.

Darüber hinaus indiziert der Rückgang der Differenz $\Delta E_p$ mit steigender Scanrate einen Spannungsabfall im System, was auf ohmsche Verluste und kinetische Einschränkungen hinweist.
Da nun mehr Strom fließen muss, um die gleiche Menge an Elektronen zu übertragen, steigt der Einfluss von Widerständen im System, das System heizt sich infolgedessen leicht auf.

Dieses Aufheizen könnte auch eine Erklärung für die Abweichung der gemessenen $\Delta E_p$-Werte von dem theoretischen Wert von \qty{59}{\milli\volt} sein, da die Nernst-Gleichung temperaturabhängig ist.

\begin{figure}[H]
    \centering
    \includegraphics[width=\textwidth]{E4/plots/10mVs.png}
    \caption{Cyclovoltammogramm bei \qty{10}{\milli\volt\per\second}.}
    \label{fig:10mVs}
\end{figure}
\begin{figure}[H]
    \centering
    \includegraphics[width=\textwidth]{E4/plots/100mVs.png}
    \caption{Cyclovoltammogramm bei \qty{100}{\milli\volt\per\second}.}
    \label{fig:100mVs}
\end{figure}
\begin{figure}[H]
    \includegraphics[width=\textwidth]{E4/plots/500mVs.png}
    \caption{Cyclovoltammogramm bei \qty{500}{\milli\volt\per\second}.}
    \label{fig:500mVs}
\end{figure}

\section{Zusammenfassung}
In diesem Versuch wurde das elektrochemische Verhalten von Ferrocen in \qty{0.1}{\mol\per\liter} TBAF/Acetonitril mittels Cyclovoltammetrie untersucht.
Die Redox-Parameter wurden bestimmt, und die Reversibilität der Reaktion wurde bedingt bestätigt.
Der Diffusionskoeffizient von Ferrocen wurde zu $D = \qty{2.14 +- 0.05e-5}{\cm\squared\per\second}$ berechnet, was sehr gut mit dem Literaturwert von $D_{lit} = \qty{2.12e-5}{\cm\squared\per\second}$\autocite{wang2010measurement} übereinstimmt.
Der Einfluss der Scanrate auf die Cyclovoltammogramme wurde analysiert, wobei höhere Scanraten zu höheren Spitzenströmen und einer Verringerung der Differenz der Oxidations- und Reduktionspotentiale führten.

\newpage
\printbibliography
\newpage

\appendix
\section{Fehlerrechnung}
\subsection{Diffusionskoeffizient}
Der Diffusionskoeffizient $D$ wurde durch eine lineare Regression der anodischen Spitzenströme $i_{pa}$ gegen die Quadratwurzel der Scanrate $v^{1/2}$ bestimmt.
Der Fehler der Steigung ist die halbe Spanne der Extremwerte.
\begin{align*}
    \Delta m &= \frac{|m_{max} - m_{min}|}{2} \\
    \Delta m &= \frac{2.8085 - 2.75}{2} \\
             &= 0.02925 \approx \qty{0.03e-6}{\ampere\per(\milli\volt\per\second)^{1/2}}
\end{align*}

Der Fehler des Diffusionskoeffizienten $D$ ergibt sich durch die Gaußsche Fehlerfortpflanzung:
\begin{align*}
    \Delta D &= \sqrt{\mleft( \pdv{D}{m} \Delta m \mright)^2} \\
    \pdv{D}{m} &= 2 \cdot \frac{m}{(2.69e5)AC^2} \\
    \Delta D &= (2 \cdot \frac{\qty{2.779e-6}{\ampere\per(\milli\volt\per\second)^{1/2}}}{(2.69e5)(0.071)(1e-4)^2} \cdot \qty{0.03e-6}{\ampere\per(\milli\volt\per\second))^{1/2}} \\
            &= \qty{0.05e-5}{\cm\squared\per\second}
\end{align*}

\section{Programmcode}
In diesem Abschnitt sei der verwendete Code nur auszugsweise diskutiert.
Triviale Funktionen wie das Einlesen von CSV-Dateien, Bestimmung von Extrema, oder das Plotten von Graphen werden hier nicht erläutert.

Um einen Bericht und Plot für eine Scanrate von beispielsweise \qty{100}{\milli\volt\per\second} zu erzeugen, wird das Skript wie folgt aufgerufen:
\begin{minted}{bash}
    \$ python3 plot 100 --plot
    (Ausgabe von Redox-Parametern)
\end{minted}

Das Programm verwendet Datenklassen, um die verschiedenen Ergebnisse besser zu strukturieren, ohne dass viele einzelne Rückgabewerte verwendet werden müssen oder Maps mit unübersichtlichen Schlüssel-Wert-Paaren.

\subsection{Aufteilung der Messdaten in Zyklen}
Zur sinnvollen Berechnung der Spitzenströme werden zunächst die einzelnen Durchläufe des Messprogramms gesucht und extrahiert.
Da unter Umständen das Signal nahe des linkseitigen Nulldurchgangs des Potentials verrauscht ist, wird zunächst eine Glättung des Potentialsignals über ein ausreichend großes Fenster durchgeführt.
Anschlißend werden die Nulldurchgänge des geglätteten Signals gesucht, und nur solche berücksichtigt, die einen Mindestabstand zueinander aufweisen.
Dieses Verfahren lief am gesamten aufgezeichneten Datensatz zuverlässig.

\newpage
\begin{minted}{python}
def find_runs(
    potentials: Sequence,
    currents: Sequence
) -> Tuple[List[List[float]], List[List[float]]]:
    pot_arr = np.asarray(potentials, dtype=float)
    n = len(pot_arr)
    if n == 0:
        return [], []

    window = 11 if n >= 11 else (5 if n >= 5 else 3)
    if window % 2 == 0:
        window += 1

    kernel = np.ones(window) / window
    smoothed = np.convolve(pot_arr, kernel, mode="same")
        candidate_crossings: List[int] = [
        i for i in range(1, n) if (smoothed[i - 1] < 0 and smoothed[i] >= 0)
    ]

    min_sep = max(5, window)
    crossings: List[int] = []
    last = -min_sep * 2
    for idx in candidate_crossings:
        if idx - last >= min_sep:
            crossings.append(idx)
            last = idx

    splits: List[Tuple[int, int]] = []
    start = 0
    for idx in crossings:
        splits.append((start, idx + 1))
        start = idx
    splits.append((start, n))

    potentials_runs = _apply_splits(potentials, splits)
    currents_runs = _apply_splits(currents, splits)
    return potentials_runs, currents_runs
\end{minted}

\newpage

Der Fehler der Mittelwerte wird als Standardfehler berechnet:
\begin{minted}{python}
def _sem(data: Sequence[float]) -> float:
    """Return standard error of the mean (s / sqrt(n)). 0.0 for n <= 1."""
    n = len(data)
    if n <= 1:
        return 0.0
    return statistics.stdev(data) / math.sqrt(n)
\end{minted}

Diese Funktion wird auch von anderen Routinen zur Fehlerberechnung verwendet.

\subsection{Basislinienkorrektur}
Um die Spitzenströme korrekt zu bestimmen, wird der kapazitive Anteil des Stroms durch eine lineare Basislinie approximiert und vom gemessenen Strom subtrahiert.
Hierzu werden die Datenpunkte in einem Bereich nach dem Potentialumkehrpunkt verwendet, um eine lineare Regression durchzuführen, unter Berücksichtigung der Sweep-Richtung.
\begin{minted}{python}
def _fit_linear_window(
    pot_run: Sequence[float],
    cur_run: Sequence[float],
    lo: float,
    hi: float,
    direction: str,
) -> Tuple[float | None, float | None, float | None, np.ndarray]:
    """Fit y = m x + b on points with lo <= x <= hi, filtered by sweep direction."""
    pot_arr = np.asarray(pot_run, dtype=float)
    cur_arr = np.asarray(cur_run, dtype=float)

    if len(pot_arr) < 2:
        return None, None, None, np.array([])

    # [...]

    m, b = np.polyfit(x_used, y_used, 1)
    y_pred = m * x_used + b
    ss_res = float(np.sum((y_used - y_pred) ** 2))
    ss_tot = float(np.sum((y_used - np.mean(y_used)) ** 2))
    r2 = 1.0 - ss_res / ss_tot if ss_tot > 0 else 1.0
    return float(m), float(b), r2, x_used
\end{minted}

Die eigentliche Korrektur findet dann durch Subtraktion der Basislinie vom gemessenen Strom statt, unter Berücksichtigung der Unsicherheiten der Regressionsparameter.
\begin{minted}{python}
def analyze_peaks(
    extremas: ExtremaStats,
    fits: Dict[str, FitWindowResult]
) -> PeakAnalysis:
    out = # [...]
    # Forward (anodic) peak vs forward fit
    fwd = fits.get("forward")
    if (
        fwd and fwd.avg.slope is not None 
        and fwd.avg.intercept is not None
        and out.i_pa_raw is not None
    ):
        m = fwd.avg.slope
        b = fwd.avg.intercept
        sigma_m = fwd.avg.slope_sem or 0.0
        sigma_b = fwd.avg.intercept_sem or 0.0
        x_pa = extremas.avg_max_pot
        fit_pa = m * x_pa + b
        out.fit_at_pa = fit_pa
        diff_pa = out.i_pa_raw - fit_pa
        out.i_pa_bg = float(np.abs(diff_pa))
        var_pa = (sigma_I_pa ** 2) + (x_pa * sigma_m) ** 2
                    + (m * sigma_x_pa) ** 2 + (sigma_b ** 2)
        out.i_pa_bg_err = float(np.sqrt(var_pa)) if var_pa > 0 else 0.0
    # [...]
    return out
\end{minted}

\subsection{Berichte und Plots}
Zur Erzeugung der Plots wird Matplotlib verwendet.
Für eine übersichtliche Darstellung der Ergebnisse als Textausgabe werden die einzelnen berechneten Resultate (Regressionen, Peak-Analysen, etc.) in tabellarischer Form ausgegeben.
Es wird für die jeweilige Scanrate ein eigener Bericht generiert, der die wichtigsten Parameter zusammenfasst.
\begin{minted}{text}
Report for scanrate 300 mV/s 
  Detected 4 runs.

=== Extrema Summary ===
  Run 1: max = 50.04 µA at 468.35 mV | min = -37.91 µA at 386.50 mV
  Run 2: max = 48.40 µA at 463.48 mV | min = -38.49 µA at 381.34 mV
  Run 3: max = 47.81 µA at 463.65 mV | min = -38.61 µA at 386.17 mV
  Run 4: max = 47.54 µA at 463.80 mV | min = -38.66 µA at 381.10 mV
  Avg Max Current: 48.45 ± 0.56 µA at 464.82 ± 1.18 mV
  Avg Min Current: -38.42 ± 0.17 µA at 383.78 ± 1.48 mV

=== Analysis Report ===
  Scanrate: 300 mV/s
  Max current potential (anodic): 464.82 ± 1.18 mV
  Min current potential (cathodic): 383.78 ± 1.48 mV
  ΔE (max - min): 81.04 ± 2.66 mV
  Halfway potential: 424.30 ± 15.34 mV
  i_pa (background-corrected): 48.254 ± 0.817 µA
  i_pc (background-corrected): 46.997 ± 0.264 µA
  i_pa / i_pc: 1.027 ± 0.018

=== Linear fit: forward ===
  Window: [0, 200] mV (dir: pos)
  Run 1: slope = 0.0051 µA/mV, intercept = -0.3061 µA, R² = 0.4524
  Run 2: slope = 0.0047 µA/mV, intercept = -2.2565 µA, R² = 0.9298
  Run 3: slope = 0.0048 µA/mV, intercept = -2.7104 µA, R² = 0.9854
  Run 4: slope = 0.0047 µA/mV, intercept = -2.8868 µA, R² = 0.9989
  Avg: slope = 0.0048 ± 0.0001 µA/mV, intercept = -2.0400 ± 0.5930 µA

=== Linear fit: reverse ===
  Window: [600, 700] mV (dir: neg)
  Run 1: slope = 0.0171 µA/mV, intercept = 2.7872 µA, R² = 0.9990
  Run 2: slope = 0.0161 µA/mV, intercept = 2.4089 µA, R² = 0.9991
  Run 3: slope = 0.0158 µA/mV, intercept = 2.2238 µA, R² = 0.9992
  Run 4: slope = 0.0156 µA/mV, intercept = 2.1011 µA, R² = 0.9988
  Avg: slope = 0.0161 ± 0.0003 µA/mV, intercept = 2.3802 ± 0.1497 µA
\end{minted}

\includepdf[pages=-]{E4/plot.pdf}
