\section{Einleitung}

In diesem Versuch wird die Oberflächenspannung verschiedener Lösungen bestimmt.
Hierzu wird die Blasendruckmethode verwendet, bei der der Druck gemessen wird, der benötigt wird, um eine Luftblase durch eine Flüssigkeit zu drücken.
Ferner wird die Temperaturabhängigkeit der Oberflächenspannung untersucht, sowie das Verhalten von kapillaraktiven und kapillarinaktiven Substanzen beobachtet.

\section{Theorie}

Die Oberflächenspannung $\sigma$ --- stellenweise auch $\gamma$ --- ist eine physikalische Größe, die die Energie beschreibt, die benötigt wird, um die Oberfläche einer Flüssigkeit zu vergrößern.
Nach der Blasendruckmethode wird eine Kapillare in die Flüssigkeit eingetaucht und eine Luftblase herausgedrückt.
Hierzu muss einerseits der hydrostatische Druck $p_h$, abhängig von der Eintauchtiefe $h_k$, aber auch der Kapillardruck $p_{\sigma}$, abhängig vom Radius der Kapillare, aufgewandt werden.
Es folgt für den Gesamtdruck $p_g$
\begin{align}
    p_h        &= h_k \rho g \nonumber \\
    p_{\sigma} &= \frac{2\sigma}{r_k} \nonumber \\
    p_g        &= \frac{2\sigma}{r_k} + \rho_p g h_k
\end{align}
Der Ausdruck für $p_{\sigma}$ leitet sich aus der Young-Laplace-Gleichung her, die eine Druckdifferenz beschreibt.
Zunächst wird vom hydrostatischen Gleichgewicht ausgegangen.
Die Grenzfläche bildet in genügend schmalen Röhrchen einen Meniskus aus, der ein Kugelsegment darstellt.
Der Radius $R$ dieser Kugel hängt ausschließlich vom Kontaktwinkel der Grenzfläche mit der Kapillare ab.
Es folgt somit für $\Delta p$
\begin{align*}
    \Delta p = \rho gh &= \frac{2\sigma\cos\theta}{a} \\
                     R &= \frac{a}{\cos\theta} \\
              \Delta p &= \frac{2\sigma}{R}\, \text{.}
\end{align*}

\newpage

Nun wird recht klar, dass der Maximaldruck erreicht ist, wenn der Radius der Blase --- anschaulich ein auf dem Kopf stehender Meniskus --- gerade dem Kapillarradius entspricht,
da die Druckdifferenz bei einem Kontaktwinkel von \qty{90}{\degree} null ist.
Für die Bestimmung der Oberflächenspannung genügt es somit, eben diesen Maximaldruck zu messen.
\begin{align}
    \sigma &= ( \rho_{H_{2}O} \cdot \Delta h - \rho_p \cdot h_k) \cdot \frac{r_k g}{2} \label{eqn:oberflächenspannung} \\
    r_k &= 2 \cdot \frac{\sigma}{(\rho_{H_{2}O} \cdot \Delta h - \rho_p \cdot h_k) \cdot g} \label{eqn:kapillarradius}
\end{align}

Die Temperaturabhängigkeit ist nach \textit{Eötvös} gegeben durch
\begin{align}
                \sigma_{mol} &= a \big( (T_k - 6 \text{\unit{\kelvin}} - T) \big) \\
    \text{mit } \sigma_{mol} &= \sigma V^{\frac{2}{3}} \nonumber
\end{align}

Kapillaraktive Stoffe neigen zur Anreicherung an der Oberfläche des Stoffes, in dem sie gelöst sind; so sammeln sich beispielsweise Tenside eher an der Wasseroberfäche.
Umgekehrt streben kapillarinaktive Stoffe von der Oberfläche weg, wie es für gelöste Salze --- also Ionen --- der Fall ist.
Diese umgeben sich mit einer möglichst großen Hydrathülle, was intuitiv an der Wasseroberfläche gehindert ist.
Betrachtet man nun das Verhältnis zwischen An- beziehungsweise Abreicherung an der Grenzfläche zur gesamten Konzentration im Volumen, so erhält man den Grenzflächenüberschuss $\Gamma_2$.
Trägt man $\sigma$ als Funktion der Konzentration $x_2$ beziehungsweise $\ln x_2$ auf, so entspricht die Ableitung am entsprechendem Punkt gerade der Grenzflächenkonzentration.
\begin{align}
    \Gamma_2 &= - \frac{1}{RT} \left( \pdv{\sigma}{\ln x_2} \right) = - \frac{x_2}{RT} \left( \pdv{\sigma}{x_2^{\prime}} \right) \label{eqn:grenzfläche}
\end{align}


\section{Durchführung}

Ein Manometer wurde bis zur Hälfte der Schenkel mit Wasser befüllt.
In einem daran angeschlossenen Vorratsgefäß wurde ebenfalls bis zur Hälfte Wasser eingefüllt.
Eine Kapillare wurde in die Probenflüssigkeit getaucht, bis der Abstandshalter gerade die Oberfläche berührt.
Die Eintauchtiefe wurde notiert.

Ein Temperiergefäß wurde mit Wasser befüllt und der Thermostat auf \qty{21}{\degreeCelsius} eingestellt.
Aus dem Vorratsgefäß wurde Wasser ausgelassen, sodass ein Druck in dem gesamten Aufbau ensteht.
Der sich aufbauende Druck wurde am Manometer mittels Millimeterpapier markiert.
Dies wurde für 30, 40, 50, 60, und \qty{70}{\degreeCelsius} wiederholt.
Jede Messung wurde in Vierfachbestimmung durchgeführt.

Die Dichte einer 5~Vol.\% Lösung von Aceton in Wasser wird bestimmt.
Dazu wurde ein \qty{50}{\milli\liter} Maßkolben leer gewogen, bis zur Eichmarke befüllt, und erneut gewogen.
In ein \qty{100}{\milli\liter} Becherglas wurden \qty{50}{\milli\liter} dieser Probenlösung gefüllt.
Die Oberflächenspannung wurde wie oben beschrieben bestimmt.
Dieser Versuchsteil wird für 10, 20, 25, 50, und 100\% Aceton wiederholt.

\section{Auswertung}

\subsection{Kapillarradius}
Zur Bestimmung des Kapillarradius wird \autoref{eqn:kapillarradius} verwendet.
$\Delta h$ wurde in vierfacher Bestimmung zu \qty{0.0338 \pm 0.001}{\meter} gemittelt.
$\sigma$ = \qty{0.0726}{\newton\per\meter} und $\rho_{H_{2}O}$ = \qty{998.1}{\kilo\gram\per\cubic\meter} wurden dem Skript\autocite{e3_skript} entnommen.
\begin{align*}
    r_k &= 2 \cdot \frac{\sigma}{(\rho_{H_{2}O} \cdot \Delta h - \rho_p \cdot h_k) \cdot g} \\
        &= 2 \cdot \frac{\qty{0.07236}{\newton\per\meter}}
            {(\qty{998.1}{\kilo\gram\per\cubic\meter} \cdot \qty{0.0338}{\meter} - \qty{998.1}{\kilo\gram\per\cubic\meter} \cdot \qty{0.015}{\meter}) \cdot \qty{9.81}{\meter\per\second\squared}} \\
        &= \qty{7.909 e-4}{\meter} \approx \qty{0.8 \pm 0.15}{\milli\meter}
\end{align*}

\subsection{Aceton-Wasser-Gemische}

\subsubsection{Dichtebestimmungen}
Der angebene Fehler der verwendeten Maßkolben beträgt \qty{\pm 0.06}{\milli\liter}.
Es wurde zusätzlich ein Ablesefehler abgeschätzt, sodass der Gesamtfehler der Maßkolben zu \qty{\pm 0.5}{\milli\liter} angenommen wird.
Die Dichte $\rho$ wird über die Massendifferenz und das Volumen $V$ berechnet. Beispielhaft für die 5~Vol.\% Lösung:
\begin{align*}
    \rho &= \frac{\Delta m}{V} \\
         &= \frac{88.93\, \text{\unit{\gram}} - 39.70\, \text{\unit{\gram}}}{50\, \text{\unit{\milli\liter}}} \\
    \rho &= 0.9846\, \text{\unit{\gram\per\milli\liter}} = 0.9846\, \text{\unit{\kilo\gram\per\liter}} \, \text{.}
\end{align*}

\begin{table}[H]
    \centering
    \caption{
        Massen der leeren Maßkolben und der befüllten Maßkolben zur Dichtebestimmung der Aceton-Wasser-Gemische.
        Die Unsicherheiten wurden nach obigem Verfahren berechnet und auf die erste von null verschiedene Stelle gerundet.
        Die berechneten Dichten wurden entsprechend der Unsicherheit gerundet.
    }
    \begin{tabular}{c c c c c c}
        \toprule
        Vol.\% & $c$ / \unit{\mol\per\liter} & $m_{\text{leer}}$ / \unit{\gram} & $m_{\text{voll}}$ / \unit{\gram} & $\Delta m$ / \unit{\gram} & $\rho$ / \unit{\gram\per\milli\liter} \\ \midrule
        5\%   & 0.68 & 39.70 \pm{} 0.01 & 88.93 \pm{} 0.01 & 49.23 \pm{} 0.02 & 0.98 \pm{} 0.01 \\ \midrule
        10\%  & 1.36 & 40.93 \pm{} 0.01 & 90.16 \pm{} 0.01 & 49.23 \pm{} 0.02 & 0.98 \pm{} 0.01 \\ \midrule
        20\%  & 2.72 & 43.82 \pm{} 0.01 & 92.25 \pm{} 0.01 & 48.43 \pm{} 0.02 & 0.97 \pm{} 0.01 \\ \midrule
        25\%  & 3.40 & 42.29 \pm{} 0.01 & 90.39 \pm{} 0.01 & 48.10 \pm{} 0.02 & 0.96 \pm{} 0.01 \\ \midrule 
        50\%  & 6.80 & 40.08 \pm{} 0.01 & 86.66 \pm{} 0.01 & 46.58 \pm{} 0.02 & 0.93 \pm{} 0.01 \\ \midrule
        100\% & 13.6 & -- & -- & -- & 0.7910 \\
        \bottomrule
    \end{tabular}
\end{table}

\subsubsection{Oberflächenspannungen}
Die Oberflächenspannung berechnet sich mit \autoref{eqn:oberflächenspannung} aus dem gemessenen $\Delta h$, der gemessenen Eintauchtiefe $h_k$ und dem berechneten Kapillarradius $r_k$.
\begin{align*}
    \sigma &= ( \rho \cdot \Delta h - \rho_p \cdot h_k) \cdot \frac{r_k g}{2} \\
    \sigma_{5\%} &= ( \qty{980}{\kilo\gram\per\cubic\meter} \cdot \qty{28}{\milli\meter} - \qty{980}{\kilo\gram\per\cubic\meter} \cdot \qty{0.015}{\meter} )
                    \cdot \frac{\qty{0.0008}{\meter} \cdot \qty{9.81}{\meter\per\second\squared}}{2} \\
                 &= 49.99
\end{align*}
\begin{table}[H]
    \centering
    \caption{Berechnete Konzentrationen, dereren natürlichen Logarithmus, und Oberflächenspannungen der untersuchten Aceton-Lösungen.}
    \begin{tabular}{c c c}
        \toprule
        $c$ / \unit{\mole\per\liter} & $\ln c$ & $\sigma$ / \unit{\milli\newton\per\meter}  \\ \midrule
        0.68 & -0.39 & 49.9 \\ \midrule
        1.36 & 0.31 & 42.3 \\ \midrule
        2.72 & 1.00 & 32.4 \\ \midrule
        3.40 & 1.22 & 26.4 \\ \midrule
        6.8 & 1.92 & 15.7 \\ \midrule
        13.6 & 2.61 & -0.62 \\
        \bottomrule
    \end{tabular}
    \label{tab:placeholder}
\end{table}

Die grafische Darstellungen sind dem Anhang beigefügt.
Der Verlauf entspricht den Erwartungen, da nach \autoref{eqn:grenzfläche} die Grenzflächenkonzentration eine Exponentialfunktion ist.
Durch Linearisierung über Auftragung der Oberflächenspannung gegen den natürlichen Logarithmus der Konzentration kann man anhand der Ableitung dessen die Grenzflächenspannung bestimmen.
Die Steigung der eingezeichneten Gerade wurde zu $y = -16.42 x + 46$ bestimmt.
Es folgt somit
\begin{align*}
    \sigma &= 16.42 \cdot e^{-c} + 46 \\
           &= 16.42 \cdot e^{-0.1} + 46 \\
           &= 60.857\, \text{.}
\end{align*}

\subsection{Oberflächenspannung Wasser}
Die Oberflächenspannung berechnet sich mit \autoref{eqn:oberflächenspannung} aus dem gemessenen $\Delta h$, der gemessenen Eintauchtiefe $h_k$ und dem berechneten Kapillarradius $r_k$.
Für die Dichten von Wasser bei 21, 30, 40, 50, 60, und \qty{70}{\degreeCelsius} wurde ein Tabellenwerk\autocite{crc:std-density:water} herangezogen.
Beispielrechnung für $T$ = \qty{30}{\degreeCelsius}.
Alle berechneten Oberflächenspannung sind in \autoref{tbl:surfacetensions_water} gelistet.
\begin{align*}
    \sigma &= ( \rho_{H_{2}O} \cdot \Delta h - \rho_p \cdot h_k) \cdot \frac{r_k g}{2} \\
    \sigma_{\qty{30}{\degreeCelsius}} &= ( \qty{995.6}{\kilo\gram\per\cubic\meter} \cdot \qty{0.033}{\meter} - \qty{995.6}{\kilo\gram\per\cubic\meter} \cdot \qty{0.015}{\meter} )
                                         \cdot \frac{\qty{0.0008}{\meter} \cdot \qty{9.81}{\meter\per\second\squared}}{2} \\
                                      &= \qty{70.32}{\milli\newton\per\meter}
\end{align*}

\begin{table}[H]
    \centering
    \caption{
        Anhand von $\Delta h$ berechnete Oberflächenspannungen $\sigma$ von Wasser bei verschiedenen Temperaturen $T$ nach obigem Verfahren.
        Die Dichten $\rho$ sind Tabellenwerte aus \cite{crc:std-density:water}.
        }
    \begin{tabular}{c c c c}
    \toprule
    $T$ / \unit{\degreeCelsius} & $\rho$ / \unit{\kilo\gram\per\cubic\meter} & $\Delta h$ / \unit{\meter} & $\sigma$ / \unit{\milli\newton\per\meter} \\ \midrule
    21 & 998.1 & 0.0338 & 73.44 \\ \midrule
    30 & 995.6 & 0.033  & 70.32 \\ \midrule
    40 & 992.2 & 0.032  & 66.19 \\ \midrule
    50 & 988.0 & 0.032  & 65.91 \\ \midrule
    60 & 983.2 & 0.031  & 61.73 \\ \midrule
    71 & 977.7 & 0.030  & 57.55 \\
    \bottomrule
    \end{tabular}
    \label{tbl:surfacetensions_water}
\end{table}

Die grafische Aufzeichnung der Oberflächenspannungen gegen die Temperatur ist dem Anhang beigefügt.

Der Verlauf ist generell linear, allerdings zeichnet sich ein Knick von \qty{40}{\degreeCelsius} bis \qty{50}{\degreeCelsius} ab.
Das entspricht nicht den Erwartungen, da nach \textit{Eötvös} der Verlauf durch
\begin{equation*}
    \sigma = \qty{0.07275}{\newton\per\meter} \cdot \big( 1 - 0.002 \cdot (T - 291 \unit{K}) \big)
\end{equation*}
beschrieben werden kann.
Es muss sich hier also um einen systematischen Fehler handeln.
In der grafischen Auswertung wurde ein $1\sigma$-Konfidenzintervall verwendet, welches eben diese Werte als Ausreißer bewertet.

\printbibliography

\newpage
\appendix

\section{Berechnung der Messunsicherheiten}
\subsection{Dichtebestimmungen}
Die maximale Messunsicherheit errechnet sich über die Rechenregel für multiplikative Größen\autocite{skript-physika}, indem die Beträge der relativen Unsicherheiten addiert werden.
Erneut beispielhaft für die 5~Vol.\% Lösung.
\begin{align*}
    \Delta \rho &= \pm \left\{ \left| \frac{0.02}{49.23} \right| + \left| \frac{0.5}{50} \right| \right\} \\
                &= \pm 1.040\, \text{\%} 
\intertext{Multiplizieren liefert die absolute Unsicherheit, was hilfreich bei der Angabe von signifikanten Stellen ist.}
    \Delta \rho &= \pm 1.040\, \text{\%} \cdot 0.9846\, \text{\unit{\gram\per\milli\liter}} \\
                &= \pm 0.01\, \text{\unit{\gram\per\milli\liter}}
\end{align*}
\subsection{Kapillarradius}

Bei der Messung von $\Delta h$ und $h_k$ handelt es sich um eine beidseitige Messung.
Der Fehler von \qty{\pm 1}{\milli\meter} des Messinstrumentes muss deshalb verdoppelt werden, also \qty{\pm 2}{\milli\meter}.
\begin{align*}
    \Delta r_k &= \pm \left\{ \frac{\qty{2}{\milli\meter}}{\qty{33.8}{\milli\meter}} + \frac{\qty{2}{\milli\meter}}{\qty{15}{\milli\meter}} \right\} \\
               &= \pm 19.2\% \implies \qty{0.8}{\milli\meter} \cdot \pm 19.2\% = \qty{0.8 \pm 0.15}{\milli\meter}\, \text{.}
\end{align*}
\subsection{Oberflächenspannung}

Ähnlich zu oben sind $\Delta h$ und $h_k$ mit einem beidseitigen Fehler behaftet. $\rho$ sei fehlerfrei, da es sich hier um Tabellenwerte mit hoher Genauigkeit handelt.
Fehlerfortpflanzung nach \textit{Gauß}.
\begin{align*}
    \Delta \sigma &= \sqrt{ \left(\left(\pdv{\sigma}{\Delta h}\right) \cdot \Delta (\Delta h)\right)^2
                          + \left(\left(\pdv{\sigma}{h_k}\right) \cdot \Delta h_k\right)^2
                          + 2 \cdot \left(\left(\pdv{\sigma}{r_k}\right) \cdot \Delta r_k\right)^2
                          }\\
                  &= \sqrt{(1 \cdot \qty{2}{\milli\meter})^2 + (1 \cdot \qty{2}{\milli\meter})^2 + 2\cdot (1 \cdot \qty{0.15}{\milli\meter})^2} \\
                  &= \sqrt{8.045} = 2.83\, \text{.}
\end{align*}

\includepdf{E3/surfacetension.pdf}
\includepdf{E3/acetone.pdf}