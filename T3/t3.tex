\section{Einleitung}

 Im vorliegenden Versuch wird das Phasendiagramm eines binären Systems aus Biphenyl und Naphthalin untersucht.
 Ziel ist es, das Fest-Flüssig-Gleichgewicht dieser beiden organischen Substanzen experimentell zu erfassen, indem für verschiedene Zusammensetzungen Abkühlkurven aufgezeichnet und ausgewertet werden.
 
 Das System Biphenyl/Naphthalin ist hierfür besonders geeignet, da sich beide Substanzen in der flüssigen Phase vollständig mischen, in der festen Phase jedoch nicht, was zur Ausbildung eines Eutektikums führt.
 Durch die experimentelle Aufnahme und Analyse der Abkühlkurven können die entsprechenden Phasengrenzen (Liquidus- und Soliduslinien) ermittelt und das vollständige Fest-Flüssig-Phasendiagramm rekonstruiert werden. 

\section{Theorie}

\subsection{Eigenschaften des Systems Biphenyl/Naphthalin}

Biphenyl und Naphthalin, zwei organische Substanzen, sind in der flüssigen Phase vollständig mischbar, bilden jedoch keine Mischkristalle im festen Zustand.\autocite{t3_skript}
Diese Unmischbarkeit im festen Zustand führt zur Herausbildung eines deutlichen Eutektikums im Phasendiagramm.
Die chemisch ähnliche Struktur der beiden Substanzen, die aus aromatischen Ringsystemen besteht, fördert intermolekulare Van-der-Waals-Kräfte in der Schmelze, während ihre unterschiedliche Geometrie (kondensierte im Vergleich zu verbundenen Ringen) die Entstehung geordneter gemeinsamer Gitterstrukturen im Festkörper verhindert.

\subsection{Grundlagen der Phasendiagramme}

Ein Phasendiagramm beschreibt, welche Aggregatzustände (fest, flüssig, gasförmig) unter bestimmten Bedingungen von Temperatur, Druck und Zusammensetzung stabil sind.
Für binäre Systeme, also Systeme mit zwei Komponenten, stellt man üblicherweise das Temperatur-Zusammensetzung-Diagramm ($T$-$X$-Diagramm) bei konstantem Druck dar.\autocite{callister2020callister}
Darin werden Grenzlinien dargestellt, die die Stabilitätsbereiche der Phasen kennzeichnen.

\subsection{Liquidus- und Soliduslinie}

Die Liquiduslinie markiert den Bereich, in dem das System vollständig flüssig ist: Oberhalb dieser Linie befindet sich das gesamte System im geschmolzenen Zustand.
Unterhalb der Soliduslinie ist das System vollständig fest. 
Der Bereich zwischen diesen beiden Linien zeigt die Koexistenz von festen und flüssigen Phasen an, was bedeutet, dass in diesem Bereich beim Abkühlen Kristallisationsprozesse stattfinden.

\subsection{Eutektischer Punkt und Erstarrungsverhalten}

Besonders wichtig ist der eutektische Punkt ($E$).
Er markiert die niedrigste Temperatur ($T_E$), bei der eine flüssige Phase im Gleichgewicht mit zwei festen Phasen existiert.\autocite{t3_skript}

Beim Abkühlen kristallisieren beide Komponenten gleichzeitig aus --- dieser Punkt ist durch einen deutlichen Temperaturhaltepunkt (Arrest) in der Abkühlkurve erkennbar.
Bei anderen Zusammensetzungen zeigen sich beim Abkühlen zwei charakteristische Stellen: ein Knickpunkt (Beginn der Kristallisation einer Komponente) und ein Haltepunkt (Eutektikum erreicht, beide Phasen kristallisieren). 

\section{Sicherheits- und Gefahrenhinweise}

\adjustbox{raise=2em}{\mbox{\stackengine{\Sstackgap}{%
\textbf{\Large Biphenyl}}{\warning}{U}{l}{F}{T}{S}}}\hfill
\ghspic{exclam}\ghspic{aqpol} \\
\ghs*{h}{315}
\ghs*{h}{319}
\ghs*{h}{335}
\ghs*{h}{410}
\ghs*{p}{273}
\ghs*{p}{280}
\ghs*{p}{302+352}
\ghs*{p}{305+351+338}
\ghs*[dots=der Entsorgung als gefährlichen Abfall]{p}{501}

\adjustbox{raise=2em}{\mbox{\stackengine{\Sstackgap}{%
\textbf{\Large Naphtalin}}{\warning}{U}{l}{F}{T}{S}}}\hfill
\ghspic{exclam}\ghspic{health}\ghspic{aqpol} \\
\ghs*{h}{228}
\ghs*{h}{302}
\ghs*{h}{351}
\ghs*{h}{410}
\ghs*{p}{210}
\ghs*{p}{273}
\ghs*{p}{308+313}
\ghs*[dots=der Entsorgung als gefährlichen Abfall]{p}{501}

\section{Durchführung}

In einem Heizbad wurden fünf Biphenyl/Naphtalin-Gemische verschiedener Zusammensetzungen auf \qty{100}{\degreeCelsius} bis zur vollständigen Schmelze erhitzt.
Die Einwaagen wurden in \autoref{tab:einwaagen-biphenyl-napthtalin} festgehalten.
Anschließend wurden die Probenröhrchen --- bereits mit Thermoelement versehen --- in Reagenzgläser gesteckt, welche wiederum einzeln in Bechergläser gestellt wurden.
Während die Gemische abkühlten, wurden in Zeitabständen von etwa einer Minute die Temperaturen aller Gemische mittels eines Mehrfachmessinstrumentes ausgelesen.
Die Temperaturen wurden über einen Zeitraum von \qty{71}{\minute} erfasst.

\begin{table}[h]
    \centering
    \caption{Stoffmengenverhältnisse und eingesetzte Einwaagen von Biphenyl und Naphtalin.}
    \begin{tabular}{c c c}
        \toprule
         Verhältnis & Einwaage Biphenyl (g)  & Einwaage Naphtalin (g) \\ \midrule
         5:5        & 3.279 & 2.725 \\ \midrule
         4:6        & 2.671 & 3.332 \\ \midrule
         3:7        & 2.043 & 3.958 \\ \midrule
         2:8        & 1.387 & 4.609 \\ \midrule
         1:9        & 0.707 & 5.291 \\ \midrule
         0:10       & 0.000 & 5.996 \\
         \bottomrule
    \end{tabular}
    \label{tab:einwaagen-biphenyl-napthtalin}
\end{table}

Nicht gelistete, ergänzende Daten für die Verhältnisse 10:0 bis 5:5 (Biphenyl:Naphtalin) wurden von Gruppe F zur Verfügung gestellt.

\section{Auswertung}

Das experimentell ermittelte Fest-Flüssig-Phasendiagramm des Systems Biphenyl/Naphthalin bestätigt die theoretischen Vorhersagen für ein ideales binäres eutektisches System.
Auf der Ordinate ist die Zusammensetzung des Gemisches aufgetragen, auf der Abzisse die Temperatur.

Die Liquidus-Linie zeigt die erwartete Schmelzpunkterniedrigung (Raoult'sches Gesetz) von beiden reinen Komponenten (Naphthalin: 81°C; Biphenyl: 71°C) in Richtung eines lokalen Minimums.
Dieses Minimum ist der Eutektische Punkt, experimentell bestimmt zu $X_E$ = 0.3 Biphenyl (Stoffmengenverhältnis) und $T_E$ = \qty{53}{\degreeCelsius}.
Bei dieser Temperatur erstarrt die Mischung eutektischer Zusammensetzung vollständig.
Gemäß der Gibbs'schen Phasenregel ($F = C – P + 1 = 0$) ist dieser Punkt invariant\autocite{caroline}($F$=0), da hier drei Phasen (Flüssigkeit, festes Naphthalin, festes Biphenyl) im Gleichgewicht stehen, was das niedrigste Schmelzintervall des gesamten Systems darstellt. 

Die Solidus-Kurve (Haltepunkte-Kurve) bildet die Grenze zum rein festen Bereich.
Während die Liquidus-Linie der Theorie folgt, zeigt die Solidus-Linie mit Werten zwischen \qty{24}{\degreeCelsius} und \qty{39}{\degreeCelsius} eine signifikante Abweichung von der theoretisch konstanten eutektischen Temperatur.
Diese Werte sind signifikant unterhalb des theoretischen $T_E$-Wertes. 
Diese Diskrepanz liegt primär in experimentellen Fehlern begründet, insbesondere in der starken Unterkühlung (Supercooling) der Schmelze vor der Kristallisationsfreisetzung.
Dieses Unterkühlen mit anschließendem Auskristallisieren ist in den Abkühlkurven daran ersichtlich, dass die Probe sich nach einiger Zeit wieder erwärmt.
Darüber hinaus weicht der ermittelte eutektische Punkt von den ideal berechneten Literaturwerten\autocite{phasendiagramme-organische-stoffe}$^,$\autocite{biphenyl-naphtalin} ($X_E$ = 0.457, $T_E$ = 41.5°C) ab.

Diese Abweichung deutet auf eine Nicht-Idealität des untersuchten Gemisches oder eine Ungenauigkeit bei der Bestimmung des kritischen Knickpunktes hin.
An einigen Stellen scheinen die Abkühlkurven unvorhersehbare Sprünge zu machen---da die Auflösung des Messgerätes auf \qty{1}{\degreeCelsius} genau ist, kann hier von systematischen Rundungsfehlern des Gerätes ausgegangen werden.
Insgesamt belegen die Daten jedoch eindeutig die Existenz eines eutektischen Gleichgewichts und die Gültigkeit der thermodynamischen Prinzipien. 

Im oberen Bereich des Phasendiagramms liegt eine homogene flüssige Phase vor. Unterhalb des eutektischen Punktes $E$ ($X$ = 0.3, $T$ = \qty{53}{\degreeCelsius}) sind Biphenyl und Naphthalin fest.
Zwischen dem Schmelzpunkt des Naphthalins (\qty{81}{\degreeCelsius}) und dem eutektischen Punkt sind festes Naphthalin und eine flüssige Phase zu beobachten.
Zwischen dem Schmelzpunkt des Biphenyls (\qty{71}{\degreeCelsius}) und der eutektischen Linie liegen festes Biphenyl und eine flüssige Phase vor.

\printbibliography

\includepdf{T3/Phasendiagramm.pdf}
\includepdf{T3/Abkuehlkurve.pdf}