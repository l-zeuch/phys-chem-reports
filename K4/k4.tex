\section{Einleitung}
In diesem Versuch wird die Auflösungsgeschwindigkeit eines Salzes in Wasser untersucht, indem die Leitfähigkeit der Lösung gemessen wird.
Die Temperaturabhängigkeit der Auflösungsgeschwindigkeit wird ebenfalls betrachtet.
Hier gilt es zu beachten, dass sich die Leitfähigkeit der Lösung ebenfalls mit der Temperatur ändert.
Zuletzt wird die Sättigungskonzentration des Salzes bestimmt und mit Literaturwerten verglichen.

\section{Theorie}
Die Diffusion ist ein passiver Transportprozess, bei dem Teilchen aufgrund ihrer thermischen Bewegung von einem Ort höherer Konzentration zu einem Ort niedrigerer Konzentration wandern.
Die Geschwindigkeit, mit der eine bestimmte Teilchenzahl $n$ pro Zeiteinheit $t$ transportiert wird, abhängig von der effektiven Oberfläche $O$, der Diffusionskonstante $D$ und dem Konzentrationsgradienten $\odv{c}{x}$.
Dieser Zusammenhang wird durch das Fick'sche Gesetz beschrieben.

Im Versuch wird die gesamte Lösung ständig gerührt, sodass nur in unmittelbarer Nähe der Salzoberfläche ein Konzentrationsgradient besteht, innerhalb der restlichen Lösung ist die Konzentration $c$ nur noch von der Zeit abhängig.
Das Vorgehen ist wichtig für das Messverfahren, da anhand des zeitlichen Verlaufs der Leitfähigkeit die Lösungsgeschwindigkeit bestimmt wird.

Diese dünne Schicht, in der ungestörte Diffusion stattfindet, wird als Diffusionsschicht bezeichnet und hat die Dicke $\delta$.
Da sich in diesem Fall die Diffusionsschicht relativ zügig auf einen stationären Zustand einstellt, ist der Konzentrationsgradient in der Diffusionsschicht näherungsweise konstant.
Hierdurch folgt, dass die angrenzende Lösung gesättigt sein muss, also die Sättigungskonzentration $c_0$ aufweist.

Zunächst wird die Abkürzung $k$ gesetzt
\begin{equation}
    k = \frac{D \cdot O}{\delta \cdot V}\, \text{.} \label{eqn:schichtdicke}
\end{equation}
Der Diffusionskoeffizient $D$ wird durch eine Beziehung nach Nernst beschrieben. Speziell für 2-2-wertige Elektrolyte gilt
\begin{align}
    D &= \frac{RT}{F^2} \cdot \frac{2 \Lambda_+ \Lambda_-}{\Lambda_+ + \Lambda_-} \nonumber \\
      &= 0.89 T \cdot \frac{(\Lambda_+ \Lambda_-) \cdot 10^{-9}}{\Lambda_+ + \Lambda_-} \unit{\cm\squared\per\second}\, \text{.} \label{eqn:diffusionskoeffizient}
\end{align}
Der Auflösungsprozess lässt sich bei Salzen durch Messung der Leitfähigkeit der Lösung verfolgen.
Bei ausreichender Verdünnung ist die spezifische Leitfähigkeit $\kappa$ der Lösung proportional zur Konzentration $c$ der gelösten Ionen im gesamten Volumen.
Somit gilt für das integrierte Geschwindigkeitsgesetz:
\begin{align}
    k &= - \frac{1}{t} \ln \frac{c_0 - c}{c_0} \\
    k &= - \frac{1}{t} \ln \frac{\kappa_0}{\kappa_0 - \kappa} \label{eqn:geschwindigkeitsgesetz}
\end{align}

\section{Durchführung}
Die Dimensionen eines Gipspresslings wurden mit einem handelsüblichen Geometrie-Dreieck gemessen.
Etwa \qty{80}{\milli\liter} gesättigte Gipslösung wurde in einen Erlenmeyerkolben gefüllt und dieser in ein Wasserbad platziert.
Weitere \qty{80}{\milli\liter} destilliertes Wasser wurden in einem Messzylinder in selbiges Wasserbad gestellt.

Das Wasserbad wurde zu Beginn auf \qty{23}{\degreeCelsius} geheizt.
Nachdem das Wasser im Messzylinder und die Gipslösung eben diese Temperatur angenommen haben, wurde mittels einer Tauchelektrode die Leitfähigkeit der Gipslösung gemessen.
Anschließend wurde der Gipspressling in ein Glasrohr auf einem Drahtgitter gelegt und die Tauchelektrode so ausgerichtet, dass sie kurz über dem Gipspressling (maximal \qty{5}{\milli\meter} nach Augenmaß) hing.
Das Wasser wurde vorsichtig in das Glasrohr gefüllt und ein computergestütztes Messprogramm mit einem Messintervall von \qty{10}{\second} wurde über \qty{10}{\minute} gestartet.

Das Wasserbad wurde nach der ersten Messung auf \qty{41}{\degreeCelsius}, und danach nochmals auf \qty{62}{\degreeCelsius} geheizt, sodass die gemessene Wassertemperatur jeweils \qty{40}{\degreeCelsius} und \qty{60}{\degreeCelsius} betrug.

\section{Ergebnisse und Diskussion}
\subsection{Spezifische Leitfähigkeit $\kappa_0$ einer gesättigten Gipslösung}
Im Rahmen der experimentellen Durchführung wurden die Sättigungsleitfähigkeiten $\kappa_0$ für die unterschiedlichen Temperaturen jeweils separat bestimmt.
Die Messwerte sind nachstehend in \autoref{tab:kappa0-messwerte} gelistet. Der Messfehler wurde auf Basis der Gerätanzeige zu \qty{\pm 50}{\micro\siemens\per\cm} geschätzt (Last-Digit-Fehler).
\begin{table}[H]
    \centering
    \caption{Gemessene spezifische Leitfähigkeit $\kappa_{0,exp}$ in \unit{\micro\siemens\per\cm} der gesättigten Gipslösung bei verschiedenen Temperaturen $T$, verglichen mit den entsprechenden Literaturwerten $\kappa_{0,lit}$\autocite{skript-k4b}.}
    \begin{tabular}{c c c c}
       \toprule
        $T\, [\unit{\degreeCelsius}]$ & $\kappa_{0,exp}$ & $\Delta \kappa_{0,exp}$ & $\kappa_{0,lit}$ \\ \midrule
        23 & 2110 & \pm{} 50 & 1920 \\ \midrule
        40 & 2860 & \pm{} 50 & 2490 \\ \midrule
        60 & 3620 & \pm{} 50 & 3200 \\
       \bottomrule
    \end{tabular}
    \label{tab:kappa0-messwerte}
\end{table}

Die experimentell bestimmten Werte für die spezifische Leitfähigkeit $\kappa_0$ liegen jeweils über den Literaturwerten.
Dies kann auf Messungenauigkeiten bei der Bestimmung der Temperatur und der Leitfähigkeit zurückzuführen sein.
Für die weitere Auswertung und die Berechnung der Löslichkeiten und Kinetikkonstanten wurden die gemessenen $\kappa_0$ verwendet.

\subsection{Bestimmung der Auflösungsgeschwindigkeit}
Aus der zeitlichen Änderung der spezifischen Leitfähigkeit während der Auflösung des Presslings lässt sich die Geschwindigkeitskonstante $k$ bestimmen.
Hierzu wird die Leitfähigkeit in Abhängigkeit von der Zeit ausgewertet und mittels \autoref{eqn:geschwindigkeitsgesetz} einer linearen Regression unterzogen.
Die Steigung der Ausgleichsgeraden entspricht direkt der Reaktionskonstanten $k$.

Zur Auswertung werden die gemessenen Werte der spezifischen Leitfähigkeit $\kappa (t)$ für verschiedene Zeitpunkte $t$ analysiert.
Auf der x-Achse wird die Zeit $t$ aufgetragen, während auf der y-Achse der Ausdruck
\[
    \ln \left( \frac{\kappa_0}{\kappa_0 - \kappa (t)} \right)
\]
dargestellt wird.

Für jede der untersuchten Temperaturen (23°C, 40°C und 60°C) wurde eine separate lineare Regression durchgeführt.
Das Konfidenzintervall wurde auf $\sigma$ gesetzt ($\approx 66\%$).
Der Betrag der gesuchte Geschwindigkeitskonstante $k$ ist entsprechend den Erwartungen für jede Temperatur unterschiedlich. 
Es ist zu beobachten, dass die Geschwindigkeitskonstante $k$ mit steigender Temperatur ebenfalls zunimmt.

Das erste Fick'sche Gesetzt beschreibt die Diffusion als temperaturabhängigen Prozess, bei dem die Beweglichkeit der Teilchen mit steigender Temperatur zunimmt.
\begin{align*}
    J &= -D \pdv{c}{x} \\
\intertext{mit}
    D &= K \frac{RT}{c}
\end{align*}
Aus diesem Zusammenhang wird deutlich, dass die Diffusionskonstante $D$ proportional zur absoluten Temperatur $T$ ist, und somit auch die Auflösungsgeschwindigkeit mit steigender Temperatur zunimmt.

\begin{table}[H]
    \centering
    \caption{Grafisch bestimmte Geschwindigkeitskonstanten $k$ der Auflösungsgeschwindigkeit für verschiedene Temperaturen $T$.}
    \begin{tabular}{c c c}
    \toprule
        $T\, [\unit{\degreeCelsius}]$ & $k\, [\unit{\per\second}]$ & $\Delta k\, [\unit{\per\second}]$ \\ \midrule
        23 & \num{258.0e-6} & \pm{} \num{2.0e-6} \\ \midrule
        40 & \num{862.5e-6} & \pm{} \num{6.2e-6} \\ \midrule
        60 & \num{19.2e-3}  & \pm{} \num{0.2e-3}\\
    \bottomrule
    \end{tabular}
\end{table}

\subsection{Die Sättigungskonzentration $c_0$ der Salzlösung}

\begin{table}[H]
    \centering
    \caption{Äquivalentleitfähigkeiten $\Lambda$ für Calcium- und Sulfat-Ionen bei verschiedenen Temperaturen $T$. Adaptiert aus der Versuchsanleitung.}
    \begin{tabular}{c c c}
        \toprule
        $T [\unit{\degreeCelsius}]$ & $\Lambda_{\ch{Ca^2+}} [\unit{\centi\meter\squared\per\ohm\per\mol}] $ & $\Lambda_{\ch{SO4^2-}} [\unit{\centi\meter\squared\per\ohm\per\mol}]$ \\ \midrule
        23 & 65.9 & 90.4 \\ \midrule
        40 & 97.0 & 127.0 \\ \midrule
        60 & 133.5 & 170.1 \\
        \bottomrule
    \end{tabular}
    \label{tab:Lambda-literatur}
\end{table}

Zur Bestimmung der Sättigungskonzentration $c_0$ wird die gemessene Sättigungsleitfähigkeit $\kappa_0$ (\autoref{tab:kappa0-messwerte}) verwendet.
Aus der Leitfähigkeit und den bekannten Äquivalentleitfähigkeiten der Ionen (\autoref{tab:Lambda-literatur}) lässt sich $c_0$ berechnen.
\begin{align}
    c_0 &= \left( \frac{\kappa_0}{\Lambda^+ + \Lambda^-} \right) \cdot 1000 \label{eqn:sättigungskonzentration} \\
    c_0(\qty{23}{\degreeCelsius}) &= \left( \frac{\qty{0.0021}{\per\ohm\per\centi\meter}}{\qty{65.9}{\centi\meter\squared\per\ohm\per\mol} + \qty{90.4}{\centi\meter\squared\per\ohm\per\mol}} \right) \cdot 1000 \nonumber \\
                                 &= \qty{0.0135}{\mol\per\liter} \nonumber
\end{align}
Analog werden $c_0$ für \qty{40}{\degreeCelsius} und \qty{60}{\degreeCelsius} berechnet.
\begin{table}[H]
    \centering
    \caption{Nach obigem Verfahren berechnete $c_0$ der Salzlösung für verschiedene Temperaturen $T$.}
    \begin{tabular}{c c c c}
    \toprule
        $T\, [\unit{\degreeCelsius}]$ & $c_0\, [\unit{\milli\mol\per\liter}]$ & $\Delta c_0\, [\unit{\milli\mol\per\liter}]$ & $c_{0}\, [\unit{\milli\mol\per\liter}]$ (Literatur) \\ \midrule
        23 & 13.5 & \pm{} 0.4 & 11.82 ($T = \qty{20.0}{\degreeCelsius}$)\autocite{Raupenstrauch1885-ng}\\ \midrule
        40 & 12.8 & \pm{} 0.3 & 12.28 ($T = \qty{38.4}{\degreeCelsius}$)\autocite{Raupenstrauch1885-ng} \\ \midrule
        60 & 11.9 & \pm{} 0.2 & 11.87 ($T = \qty{59.0}{\degreeCelsius}$)\autocite{Raupenstrauch1885-ng} \\
    \bottomrule
    \end{tabular}
\end{table}
Die Diskrepanz zwischen den theoretisch erwarteten und den experimentell gemessenen Löslichkeitswerten $c_0$ kann sich aus Messungenauigkeiten bei der Bestimmung der Sättigungsleitfähigkeit $\kappa_0$ ergeben.
So konnte die Temperatur des Wasserbades nicht konstant auf dem Sollwert gehalten werden, was zu Abweichungen bei der Leitfähigkeitsmessung führen kann.
Der Gerätefehler des Leitfähigkeitsmessgeräts von \qty{\pm 0.1}{\micro\siemens\per\cm} trägt ebenfalls zur Unsicherheit bei, obwohl dieser im Vergleich zur Temperatur eher weniger ins Gewicht fällt.
Ferner ist zu beachten, dass die Literaturwerte bei leicht abweichenden Temperaturen angegeben sind.
Diese Faktoren führen dazu, dass die experimentellen $c_0$-Werte höher als die angegebenen Literaturwerte ausfallen.

Die Abnahme der Löslichkeit mit steigender Temperatur deutet auf eine retrograde Löslichkeit hin, was auf einen exothermen Auflösungsprozess schließen lässt.
Dieser ist bei Gips bekannt, mit einem Maximum der Löslichkeit bei etwa \qty{38}{\degreeCelsius}.\autocite{partridge1929solubility}
Dieses Maximum ist jedoch in den experimentellen Daten nicht erkennbar, was auf systematische Fehler in der Versuchsdurchführung hindeutet, speziell bei $T = \qty{23}{\degreeCelsius}$.

\subsection{Experimentelle Ermittlung der Diffusionsschichtdicke}

\subsubsection{Oberfläche des Presslings}
Der Pressling wird genähert als Zylinder betrachtet.
Aufgrund des Lösungsvorganges ist jedoch davon auszugehen, dass die tatsächliche Oberfläche größer als genähert ist, und sich zusätzlich mit der Zeit verändert.
Da Calciumsulfat jedoch schwer wasserlöslich ist und somit nur kleinste Veränderungen der Oberfläche zu erwarten sind, ist diese Näherung gerechtfertigt.
\begin{align}
    O_{Zylinder} &= \pi dh + 2\pi r^2 \label{eqn:oberfläche}\\
    d &= \qty{1.3 \pm 0.1}{\milli\meter} \nonumber \\
    h &= \qty{0.2 \pm 0.1}{\centi\meter} \nonumber \\
    \implies O_{Zylinder} &= \pi \cdot \qty{1.3 \pm 0.1}{\centi\meter} \cdot \qty{0.2 \pm 0.1}{\centi\meter} + \pi \cdot (\qty{1.3\pm 0.1}{\centi\meter})^2 \nonumber \\
    O &= \qty{6.12 \pm 0.98}{\centi\meter\squared}\, \text{.} \nonumber
\end{align}

\subsubsection{Berechnung der Schichtdicke $\delta$}

Die Schichtdicke berechnet sich nach \autoref{eqn:schichtdicke}.
Hierzu wird zunächst das Verhältnis $\frac{O}{\delta}$ aus den Gleichgewichtskonstanten $k$ und Diffusionskoeffizienten $D$, sowie des Volumens $V = \qty{80}{\milli\liter}$ bestimmt.
\begin{align*}
    \frac{O}{\delta} &= \frac{k}{D}V := X \\
    \delta &= \frac{D \cdot O}{k \cdot V} \tag{\ref{eqn:schichtdicke} Wdh.} \\
    D &= 0.89 T \cdot \frac{(\Lambda_+ \Lambda_-) \cdot 10^{-9}}{\Lambda_+ + \Lambda_-} \unit{\cm\squared\per\second} \tag{\ref{eqn:diffusionskoeffizient} Wdh.} \\
\intertext{Für \qty{23}{\degreeCelsius} folgt somit}
    D(\qty{23}{\degreeCelsius}) &= 0.89 \cdot 296.15 \cdot \frac{(65.9 \cdot 90.4) \cdot 10^{-9}}{65.9 + 90.4}\unit{\cm\squared\per\second} \\
                                &= \qty{1.005e-5}{\centi\meter\squared\per\second} \\
               \frac{O}{\delta} &= \frac{\qty{258.0 \pm 2.0e-6}{\per\second}}{\qty{1.005e-5}{\cm\squared\per\second}} \cdot \qty{80}{\cm\cubed} \\
                                &= \qty{2053.7 \pm 4.0}{\cm} \\
                         \delta &= \frac{\qty{6.12 \pm 0.98}{\cm\squared}}{\qty{2053.7 \pm 4.0}{\cm}} \\
                                &= \qty{3.0 \pm 5.0}{\micro\meter}
\end{align*}
\begin{table}[H]
    \centering
    \caption{Berechnete Diffusionskoeffizienten $D$, Schichtdicken der Diffusionsschicht $\delta$ bei verschiedenen Temperaturen $T$.}
    \begin{tabular}{c c c c}
        \toprule
        $T\, [\unit{\degreeCelsius}]$ & $D\, [\unit{\cm\squared\per\second}]$ & $\frac{O}{\delta}\, [\unit{\cm}]$ & $\delta\, [\unit{\micro\meter}]$ \\ \midrule
        23 & \num{1.005e-5} & \num{2053.7 +- 4.0} & \num{3.0 +- 5.0} \\ \midrule
        40 & \num{1.532e-5} & \num{4503.9 +- 5.7} & \num{1.0 +- 2.2} \\ \midrule
        60 & \num{2.213e-5} & \num{69408 +- 27}   & \num{0.88 +- 0.15} \\ \midrule
    \end{tabular}
\end{table}

\subsection{Zusammenfassung}
Die Auflösungsgeschwindigkeit von Gips in Wasser wurde experimentell durch Messung der Leitfähigkeit der Lösung bei den Temperaturen \qty{23}{\degreeCelsius}, \qty{40}{\degreeCelsius} und \qty{60}{\degreeCelsius} untersucht.
Die Gleichgewichtskonstanten $k$ der Auflösung konnten aus den Messdaten grafisch bestimmt werden und zeigen eine deutliche Zunahme mit steigender Temperatur, was auf die erhöhte Diffusionsbeweglichkeit der Ionen zurückzuführen ist.
Die experimentell bestimmten Sättigungskonzentrationen der Salzlösung liegen in der Größenordnung der Literaturwerte\autocite{Raupenstrauch1885-ng}$^,$\autocite{partridge1929solubility}, zeigen jedoch Abweichungen, die auf Messungenauigkeiten und systematische Fehler zurückzuführen sind.
Der exotherme Charakter des Auflösungsprozesses von Gips konnte durch die Abnahme der Löslichkeit mit steigender Temperatur bestätigt werden.
Die berechneten Diffusionsschichtdicken liegen im Bereich von einigen Mikrometern, und nehmen mit steigender Temperatur ab, da die Diffusionsvorgänge bei höheren Temperaturen begünstigt werden.

\section{Zusatzfragen}
Der spezifische Widerstand ($\rho$) und die spezifische Leitfähigkeit ($\kappa$) sind reziproke Kenngrößen eines Materials.
$\rho$ bezeichnet die Strombehinderung und $\kappa$ die Stromleitung, wobei $\kappa$ typischerweise in \unit{\siemens\per\centi\meter} angegeben wird und sich auf eine definierte Geometrie
mit der Länge $l$ und der Querschnittsfläche $A$ bezieht.
\begin{align}
    \rho &= R \cdot \frac{A}{l} \\
    \kappa &= \frac{1}{\rho} \\
           &= \frac{L}{R \cdot A}\, \text{.}
\end{align}
Die experimentelle Bestimmung von $\kappa$ erfolgt mittels der Zellkonstante $C_{\text{Zelle}}$, die das Verhältnis von Elektrodenabstand $L$ zur Elektrodenfläche $A$ angibt.
Somit errechnet sich die spezifische Leitfähigkeit aus der gemessenen Leitfähigkeit $G$ und der Zellkonstante.\autocite{jander/blasius2}
\begin{equation}
    \kappa = G \cdot C_{\text{Zelle}}
\end{equation}
Um die Leitfähigkeit unabhängig zur Konzentration zu bewerten, wird die Äquivalentleitfähigkeit $\Lambda$ verwendet\autocite{job-ruffler-phys-chem:equivalent}, welche $\kappa$ zur Äquivalentkonzentration ($c_{\text{eq}}$) in Beziehung setzt
\begin{equation}
    \Lambda = \frac{\kappa}{c_{eq}}\, \text{.}
\end{equation}
Die gesamte Leitfähigkeit resultiert dabei aus der Wanderungsgeschwindigkeit der Ionen $v$ in der Lösung unter Einfluss eines elektrischen Feldes $E$.
$v$ wird durch die Ionenbeweglichkeit $u$ und die Feldstärke bestimmt.\autocite{job-ruffler-phys-chem:wanderungsgeschwindigkeit}
\begin{equation}
    v = u \cdot E
\end{equation}
Leitfähigkeitsmessgeräte (Konduktometer) bestimmen die spezifische Leitfähigkeit ($\kappa$) einer Elektrolytlösung, indem sie in der Messzelle den elektrischen Widerstand der Lösung zwischen inerten Elektroden messen, wobei zur Vermeidung von Polarisation Wechselstrom eingesetzt wird\autocite{job-ruffler-phys-chem:konduktometer}.

Das Messgerät berechnet $\kappa$ aus der gemessenen Leitfähigkeit und der Zellkonstante unter automatischer Temperaturkompensation auf eine Bezugstemperatur (z.B. \qty{25}{\degreeCelsius}), da $\kappa$ ---wie im Versuch bestätigt---von der Temperatur abhängt.
Die Leitfähigkeit wird primär von der Konzentration ($c$) und der Ionenbeweglichkeit ($u$) bestimmt\autocite{atkins-leitfähigkeit}, wobei $u$ von der Ionennatur (Größe, Ladung, Solvatationsgrad) und der Lösungsmittelviskosität beeinflusst wird.
Höhere Temperatur und Dissoziationsgrad ($\alpha$, insbesondere bei schwachen Elektrolyten) erhöhen $\kappa$\autocite{atkins-beweglichkeit}.

Die spezifische Leitfähigkeit ($\kappa$) steigt näherungsweise linear mit der Konzentration an, während die Äquivalentleitfähigkeit ($\Lambda$) --- welche $\kappa$ zur Konzentration in Beziehung setzt --- mit abnehmender Konzentration zunimmt, da die störenden interionischen Wechselwirkungen nachlassen.
Nach dem Gesetz von der unabhängigen Ionenwanderung (Kohlrausch) ist die Grenzleitfähigkeit ($\Lambda_0$) bei unendlicher Verdünnung die additive Summe der Grenz-Ionenleitfähigkeiten der einzelnen Ionen, da dann interionische Kräfte vernachlässigbar sind.
\begin{equation}
\Lambda_0 = \nu_+ \cdot \lambda_{0,+} + \nu_- \cdot \lambda_{0,-}
\end{equation}

\newpage
\printbibliography
\newpage
\appendix

\section{Berechnung der Messunsicherheiten}
\subsection{Geschwindigkeitskonstante}
Die Unsicherheit in $\ln \left( \frac{\kappa_0}{\kappa_0 - \kappa} \right)$ wird nach Gauß bestimmt. Die Zeit $t$ wird als fehlerfrei angenommen, da die Messung digital erfolgte.
Auf Basis der Geräteanzeige wurde hier der Fehler von $\kappa$ zu \qty{0.0000001}{\siemens\per\cm} (\qty{0.1}{\micro\siemens\per\cm}) geschätzt (Last-Digit-Fehler).
Der Lesbarkeit halber wird hier $u$ für den Logarithmus-Ausdruck verwendet.
\begin{align*}
    k &= - \frac{1}{t} \ln \frac{\kappa_0}{\kappa_0 - \kappa} \tag{\ref{eqn:geschwindigkeitsgesetz} Wdh.} \\
    \implies \Delta u &= \sqrt{\left( \pdv{k}{\kappa} \cdot \Delta \kappa \right)^2 + \left( \pdv{k}{\kappa_0} \cdot \Delta \kappa_0 \right)^2} \\
    \implies \Delta u &= \sqrt{\left( \frac{1}{(\kappa_0 - \kappa)^2} \cdot \Delta \kappa \right)^2 + \left( \frac{\kappa_0}{(\kappa_0 - \kappa)^2} \cdot \Delta \kappa_0 \right)^2} \\
    \intertext{Beispielhaft für \qty{60}{\degreeCelsius} bei $t = \qty{10}{\second}$}
    \Delta u &= \sqrt{\left( \frac{1}{(0.0032 - \num{51.4e-6})} \cdot \num{0.1e-6} \right)^2
                            + \left( \frac{0.0032}{(0.0032 - \num{51.4e-6})^2} \cdot \num{0.1e-6} \right)^2 } \\
    \Delta u &= 0.010
\end{align*}
Analog hierzu wurden für alle erfassten Werte über alle Temperaturen die Fehler mittels eines Tabellenkalkulationsprogramms berechnet und in die angehängten grafischen Auftragungen eingezeichnet.

Die Unsicherheit der Geschwindigkeitskonstanten $\Delta k$ ergibt sich aus dem Mittelwert der eingezeichneten Extremgeraden, welcher auch zugleich die Steigung der Regression ist.
Diese sind in den angehängten Aufzeichnungen direkt eingetragen.
\newpage
\subsection{Sättigungskonzentration der Salzlösung}
Die Unsicherheit in $c_0$ wird hauptsächlich durch den Fehler von $\kappa_0$ bestimmt.
\begin{align*}
    c_0 &= \left( \frac{\kappa_0}{\Lambda^+ + \Lambda^-} \right) \cdot 1000 \tag{\ref{eqn:sättigungskonzentration} Wdh.}\\
\intertext{mit}
                    \Delta c_0 &= \pdv{c_0}{\kappa_0} \cdot \Delta \kappa_0 \\
           \implies \Delta c_0 &= \left( \frac{1}{\Lambda^+ + \Lambda^-} \right) \cdot \Delta \kappa_0 \cdot 1000\, \text{.} \\
\intertext{Der Gerätefehler $\Delta \kappa_0$ wurde geschätzt zu}
    \Delta \kappa_0 &= \qty{0.00005}{\per\ohm\per\centi\meter}\, (\hat{=} \qty{50}{\micro\siemens}) \text{.}
\intertext{Für \qty{40}{\degreeCelsius} folgt}
    \Delta c_0 &= \frac{1}{\qty{224}{\centi\meter\squared\per\ohm\per\mol}} \cdot \qty{0.00005}{\per\ohm\per\centi\meter} \cdot 1000 \\
               &= \qty{0.00022}{\mol\per\liter} \\
    \Delta c_0 &\approx \qty{0.0003}{\mol\per\liter}
\end{align*}
Analog werden die $\Delta c_0$ der Temperaturen \qty{23}{\degreeCelsius} und \qty{60}{\degreeCelsius} berechnet.

\subsection{Berechnung der Schichtdicke}
\subsubsection{Oberfläche}
Die Unsicherheit in $O$ wird durch die Unsicherheiten in $d$ und $h$ bestimmt.
Fehlerrechnung nach Gauß.
\begin{align*}
    O &= \pi dh + \pi d^2 \tag{\ref{eqn:oberfläche} Wdh.}\\
    \Delta O &= \sqrt{\left( \pdv{O}{d} \cdot \Delta d \right)^2 + \left( \pdv{O}{h} \cdot \Delta h \right)^2} \\
             &= \sqrt{\left( \pi(2d + h) \Delta d \right)^2 + \left( \pi d \Delta h \right)^2 } \\
             &= \sqrt{\left( \pi (2 \cdot \qty{1.3}{\centi\meter} + \qty{0.2}{\centi\meter}) \cdot \qty{0.1}{\centi\meter} \right)^2 + \left( \pi \cdot \qty{1.3}{\centi\meter} \cdot \qty{0.1}{\centi\meter} \right)^2 } \\
    \Delta O &= \qty{0.9698}{\centi\meter\squared} \\
\end{align*}
\newpage
\subsubsection{Verhältnis O zu $\delta$}
Das Verhältnis $\frac{O}{\delta} = a$ wird aus der Gleichgewichtskonstanten $k$, dem Diffusionskoeffizienten $D$ und dem Volumen $V$ berechnet.
Von einem Fehler sei lediglich $k$ behaftet.
Es folgt nach Gauß
\begin{align*}
    \Delta a &= \sqrt{ \left( \pdv{a}{k} \cdot \Delta k \right) } \\
             &= \sqrt{ \left( \frac{V}{D} \cdot \Delta k \right) }\, \text{.} \\
\intertext{Für \qty{23}{\degreeCelsius} folgt}
             &= \sqrt{ \left( \frac{\qty{80}{\milli\liter}}{\qty{1.005e-5}{\cm\squared\per\second}} \cdot \qty{2.0e-6}{\per\second} \right) } \\
             &= \qty{3.99}{\cm} \approx \qty{4.0}{\cm}
\end{align*}
Analog werden die Unsicherheiten der Verhältnisse für \qty{40}{\degreeCelsius} und \qty{60}{\degreeCelsius} berechnet.

\subsubsection{Schichtdicke}
Die Schichtdicke hängt nun ausschließlich von $\frac{O}{\delta} = X$ und $O$ ab.
Erneut nach Gauß
\begin{align*}
    \Delta \delta &= \sqrt{ \left( \pdv{\delta}{O} \cdot \Delta O \right)^2 + \left( \pdv{\delta}{X} \cdot \Delta X \right)^2  } \\
                  &= \sqrt{ \left( \frac{1}{X} \cdot \Delta O \right)^2 + \left( - \frac{O}{X^2} \Delta X \right)^2 }\, \text{.} \\
\intertext{Für \qty{23}{\degreeCelsius} folgt}
    \Delta \delta &= \sqrt{ \left( \frac{1}{\qty{2053.7}{\cm}} \cdot \qty{0.98}{\cm\squared} \right)^2 
                            + \left( \frac{\qty{6.12}{\cm\squared}}{(\qty{2053.7}{\cm})^2} \cdot \qty{4.0}{\cm} \right)^2 } \\
                  &= \qty{4.77e-4}{\cm} \approx \qty{5}{\micro\meter}\, \text{.}
\end{align*}

\includepdf[landscape=true]{K4/23.pdf}
\includepdf[landscape=true]{K4/40.pdf}
\includepdf[landscape=true]{K4/60.pdf}
