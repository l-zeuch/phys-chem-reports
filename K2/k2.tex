\section{Einleitung}
Malachitgrün ist ein Triphenylmethan-Farbstoff, der in der Textilindustrie als Farbstoff und in der Mikrobiologie als biologischer Farbstoff verwendet wird.
In wässriger Lösung reagiert Malachitgrün mit Wasser und Hydroxid-Ionen zu einer farblosen Carbinolbase.
Die Geschwindigkeit dieser Reaktion hängt von der Konzentration der Hydroxid-Ionen ab.
Ziel dieses Versuchs ist es, die Geschwindigkeitskonstanten der Reaktionen von Malachitgrün mit Wasser und Hydroxid-Ionen zu bestimmen.
Hierzu wird die Reaktion spektrophotometrisch verfolgt.

\section{Theorie}
In wässriger Lösung reagiert das Kation des Malachitgrüns $A$ mit Hydroxid-Ionen zur entsprechenden Carbinolbase $B$.
Bei entsprechend großer Konzentration liegt das Gleichgewicht auf der Seite der Carbinolbase, es gilt also
\begin{reactions}
    A + OH- &<=>>[ $k_1$ ] B
\intertext{zusätzlich reagiert Malachitgrün mit Wasser}
    A + H2O &<=>>[ $k_2$ ] B + H+
\end{reactions}
Hierbei seien $k_1$ und $k_2$ die Geschwindigkeitskonstanten der oben angeführten Reaktionen.

Da in verdünnter Lösung gearbeitet wird, kann die Konzentration des Wassers konstant gesetzt und somit in $k_2$ einbezogen werden.
Um die Konzentrationen der Hydroxid-Ionen ähnlich konstant halten zu können, wird der pH-Wert der Lösung größer 10 gehalten.
Es folgt für den Zerfall des Malachitgrüns
\begin{align}
                - \odv{c}{t} &= k_1 c_{\ch{OH-}} \cdot c + k_2 c \nonumber \\
                           k &= k_1 c_{\ch{OH-}} + k_2 \label{eqn:geschwindigkeitskonstanten} \\
    \Rightarrow - \odv{c}{t} &= k c \nonumber
\intertext{Integration dieses Geschwindigkeitsgesetzes liefert}
           \ln \frac{c}{c_0} &= -k ( t - t_0 ) \nonumber \\
\intertext{Für die Einheiten der Geschwindigkeitskonstanten gilt}
                         [k] &= \unit{\per\second} \\
                       [k_1] &= \unit{\liter\per\mol\per\second} \nonumber \\
                       [k_2] &= \unit{\per\second} \nonumber
\end{align}
Die Solvolyse des Malachitgrüns wird spektrophotometrisch verfolgt, da die resultierende Carbinolbase farblos ist.
Durch Anwenden des Lambert-Beerschen Gesetzes wird erhalten
\begin{equation}
    \ln \frac{E}{E_0} = - k( t - t_0 ) \label{eqn:absorption}
\end{equation}
Dies ist allerdings nur gültig, wenn $A$ und $B$ nicht bei der gleichen Wellenlänge absorbieren.
Um das zu bestätigen, werden während der Reaktion etwa alle \qty{10}{\minute} Spektren aufgenommen.

Der Punkt, an dem sich alle aufgenommenen Spektren kreuzen, wird \textbf{isosbestischer Punkt} genannt.
Seine Existenz ist notwendig, um nachzuweisen, dass aus dem Ausgangsstoff $A$ nur ein Endprodukt $B$ entsteht.
Würde $B$ zu $C$ weiterreagieren, oder $A$ in einer zeitgleichen Reaktion zu $C$ reagieren, so wird kein solcher Punkt ersichtlich.

Anhand von \autoref{eqn:absorption} erhält man $k$ für eine bestimmte Hydroxid-Konzentration.
Werden mehrere Konzentration gemessen und die gefundenen $k$ gegen die Konzentration aufgetragen, so lassen sich $k_1$ und $k_2$ nach \autoref{eqn:geschwindigkeitskonstanten} bestimmen.
Dabei ist dann $k_1$ die Steigung der Ausgleichsgeraden und $k_2$ der y-Achsenabschnitt.

\section{Durchführung}
\subsection{Reagenzien}
\begin{itemize}
    \item \qty{3e-5}{\mol\per\liter} Malachitgrün in Wasser
    \item vorgefertigte Pufferlösungen mit pH-Werten von \num{10.0}, \num{10.3}, \num{10.6}, \num{11.0}, und \num{11.3}
\end{itemize}

Es wurden \qty{2}{\milli\liter} einer Lösung von \qty{3e-5}{\mol\per\liter} Malachitgrün mit \qty{2}{\milli\liter} Wasser vermischt.
Diese Lösung wurde in eine Quarzküvette gefüllt und die Extinktion mit einem Spektralphotometer aufgenommen.
Die Raumtemperatur wurde gemessen.

Es wurden weitere \qty{2}{\milli\liter} der Malachitgrün-Lösung mit \qty{2}{\milli\liter} einer Pufferlösung von pH = 10.5 vermischt.
Die Probe wurde sofort nach dem Ansetzen vermessen, dann 5 Mal in Abständen von \qty{10}{\minute}.

Das Photometer wurde auf die im ersten Versuchsabschnitt bestimmte Wellenlänge des längstwelligen Absorptionsmaximums von \qty{617}{\nano\meter} fest eingestellt.
Es wurden \qty{2}{\milli\liter} der Malachitgrün-Lösung mit \qty{2}{\milli\liter} verschiedener Pufferlösungen mit pH-Werten von \num{10.0}, \num{10.3}, \num{10.6}, \num{11.0}, und \num{11.3} vermischt.
Die Probe wurde sofort bei dieser Wellenlänge gemessen.
Die pH-Werte der Pufferlösungen wurden mit einem pH-Meter überprüft.

\section{Ergebnisse und Diskussion}
Die Raumtemperatur während des Versuchs betrug \qty{21}{\degreeCelsius} (\qty{294.15}{\kelvin}).
Die gemessenen pH-Werte der Pufferlösungen sind nachstehend in \autoref{tab:ph} dargestellt.

Hierbei sei bereits an dieser Stelle angemerkt, dass die Abweichungen insbesondere bei höheren pH-Werten auf systematische Fehler der pH-Elektrode zurückzuführen sind.
So musste, um einen stabilen Messwert zu erhalten, die Elektrode für längere Zeit in der Lösung belassen werden, was zu einer Verfälschung des Wertes führen kann.
\begin{table}[H]
    \centering
    \caption{Gewünschte und gemessene pH-Werte der Pufferlösungen.}
    \label{tab:ph}
    \begin{tabular}{c c c}
        \toprule
        Gewünschter pH-Wert & Gemessener pH-Wert \\
        \midrule
        \num{10.0} & \num{10.073 +- 0.002} \\ \midrule
        \num{10.3} & \num{10.332 +- 0.002} \\ \midrule
        \num{10.6} & \num{10.513 +- 0.002} \\ \midrule
        \num{11.0} & \num{10.775 +- 0.002} \\ \midrule
        \num{11.3} & \num{10.747 +- 0.002} \\
        \bottomrule
    \end{tabular}
\end{table}

\subsection{Absorptionsmaximum und Extinktionskoeffizient}
Das Absorptionsspektrum von Malachitgrün in Wasser ist in \autoref{fig:wasser} dargestellt.
Das längstwelligste Absorptionsmaximum liegt bei \qty{617}{\nano\meter} mit einer Extinktion von \num{0.481}.
Der zugehörige dekadische Extinktionskoeffizient $\varepsilon_{617}$ berechnet sich nach dem Lambert-Beerschen Gesetz zu
\begin{align*}
    E &= \varepsilon_{617} \cdot c \cdot d \\
    \varepsilon_{617} &= \frac{E}{c \cdot d} \\
    &= \frac{0.481}{\qty{1.5e-5}{\mol\per\liter} \cdot \qty{1}{\centi\meter}} \\
    &= \qty{32066}{\liter\per\mol\per\centi\meter}
\end{align*}

\begin{figure}[H]
    \centering
    \includegraphics[width=\textwidth]{K2/plots/Wasser.png}
    \caption{
        Absorptionsspektrum von Malachitgrün in Wasser im Bereich \qty{200}{\nano\meter} bis \qty{800}{\nano\meter}.
        Das längstwelligste Absorptionsmaximum liegt bei \qty{617}{\nano\meter}.
    }
    \label{fig:wasser}
\end{figure}

\subsection{Bestimmung des isosbestischen Punktes}
Der isosbestische Punkt ist dadurch ausgezeichnet, dass sich alle aufgenommenen Spektren während der Reaktion bei einer bestimmten Wellenlänge schneiden.
In \autoref{fig:isosbestisch} sind zwei mögliche Kandidaten ersichtlich, nämlich bei \qty{279}{\nano\meter} und im Bereich von \qty{240}{\nano\meter} bis \qty{360}{\nano\meter}.
Da sich letzterer Bereich allerdings nicht in einem einzigen Punkt schneidet, sondern sich die Schnittpunkte über einen Bereich erstrecken, und die Spektren sich nicht kreuzen, wird der isosbestische Punkt bei \qty{279}{\nano\meter} festgesetzt.

\begin{figure}[H]
    \centering
    \includegraphics[width=\textwidth]{K2/plots/isosbestisch.png}
    \caption{
        Absorptionsspektren der Probelösung während der Reaktion mit Hydroxid-Ionen bei pH = 10.5 im Bereich \qty{220}{\nano\meter} bis \qty{800}{\nano\meter}.
        Die Spektren wurden in \qty{10}{\minute} Abständen aufgenommen.
        Der isosbestische Punkt liegt bei \qty{279}{\nano\meter}.
    }
    \label{fig:isosbestisch}
\end{figure}

\subsection{Bestimmung der Geschwindigkeitskonstanten}
Zur Bestimmung der Geschwindigkeitskonstanten wird \autoref{eqn:absorption} verwendet.
$E_0$ ist hierbei die Extinktion zum Zeitpunkt $t_0 = 0$, also unmittelbar nach dem Ansetzen der Reaktion beziehungsweise dem Start der Messung.

\begin{align*}
    \ln \frac{E}{E_0} &= -k ( t - t_0 ) \tag{\ref{eqn:absorption} Wdh.} \\
    t_0 &= 0 \\
    \Rightarrow k &= - \frac{\ln \frac{E}{E_0}}{t}
    \intertext{Für pH = 10.0 (10.073) ergibt sich somit}
    k_{10.0} &= - \frac{\ln \frac{0.5716}{0.6264}}{\qty{300}{\second}} \\
    &= \qty{3.05e-4}{\per\second}
\end{align*}

Interessant ist anzumerken, dass die Geschwindigkeitskonstante bei pH = 10.747 (11.3) größer ist als bei pH = 10.775 (11.0).
Das entspricht in dieser Form nicht den Erwartungen, da mit steigender Hydroxid-Konzentration auch die Geschwindigkeitskonstante steigen sollte.
Diese Abweichung ist vermutlich auf systematische Fehler der pH-Messung zurückzuführen, die bereits zuvor angesprochen wurden.
Abgesehen davon zeigen die Daten in \autoref{tab:k} das erwartete Verhalten.

\begin{table}[H]
    \centering
    \caption{Bestimmte Geschwindigkeitskonstanten $k$ bei verschiedenen pH-Werten.}
    \label{tab:k}
    \begin{tabular}{c c c c}
        \toprule
        pH & E & $E_0$ & $k$ [\unit{\per\second}] \\
        \midrule
        \num{10.073 +- 0.002} & \num{0.5716} & \num{0.6264} & \num{30.5 +- 2.5e-5} \\ \midrule
        \num{10.332 +- 0.002} & \num{0.5419} & \num{0.6016} & \num{44.8 +- 2.5e-5} \\ \midrule
        \num{10.513 +- 0.002} & \num{0.5548} & \num{0.6339} & \num{44.4 +- 2.5e-5} \\ \midrule
        \num{10.747 +- 0.002} & \num{0.4958} & \num{0.5928} & \num{69.5 +- 2.5e-5} \\ \midrule
        \num{10.775 +- 0.002} & \num{0.5245} & \num{0.6239} & \num{57.8 +- 2.5e-5} \\
        \bottomrule
    \end{tabular}
\end{table}

Mittels einer linearen Regression nach \autoref{eqn:geschwindigkeitskonstanten} können nun $k_1$ und $k_2$ bestimmt werden.
Die Auftragung ist dem Anhang zu entnehmen.
$k_1$ entspricht hierbei der Steigung der Ausgleichsgeraden und $k_2$ dem y-Achsenabschnitt.
Für $k_1$ wurde \qty{0.70 +- 0.17}{\liter\per\mol\per\second} und für $k_2$ \qty{21.0 +- 6.4e-5}{\per\second} bestimmt.

Diese Werte liegen nahe am Literaturwert von \qty{1.267}{\liter\per\mol\per\second} für $k_1$ und \qty{25e-5}{\per\second} für $k_2$ \autocite{chen1959pressure} (bei \qty{20}{\degreeCelsius}).
Die Abweichungen sind durch Temperaturunterschiede zur Literatur, sowie des anderen Experimentaufbaus erklärbar, da in diesem Versuch statt einer Natriumhydroxid-Lösung Pufferlösungen verwendet wurden.

\section{Zusammenfassung}
In diesem Versuch wurde die Solvolyse von Malachitgrün in wässriger Lösung untersucht.
Das Absorptionsspektrum von Malachitgrün in Wasser wurde aufgenommen und das längstwelligste Absorptionsmaximum bei \qty{617}{\nano\meter} bestimmt.
Der dekadische Extinktionskoeffizient $\varepsilon_{617}$ wurde zu \qty{32066}{\liter\per\mol\per\centi\meter} berechnet.
Die Geschwindigkeitskonstanten $k$ bei verschiedenen pH-Werten wurden bestimmt und mittels linearer Regression $k_1$ und $k_2$ der Reaktionen mit Hydroxid-Ionen beziehungsweise Wasser bestimmt.

\section{Zusatzfragen}
\subsection{Bau eines UV-Spektralphotometers}
Ein UV-Spektralphotometer besteht im Wesentlichen aus einer Lichtquelle, einem Monochromator, einer Küvette zur Probenaufnahme, und einem Detektor.
Die Lichtquelle emittiert Licht im UV-Bereich, welches durch den Monochromator in seine spektralen Komponenten zerlegt wird.

Durch Einstellen des Monochromators kann eine bestimmte Wellenlänge ausgewählt werden, die dann durch die Probe in der Küvette geleitet wird.
Der Detektor misst die Intensität des durch die Probe hindurchtretenden Lichts.
Anhand der gemessenen Intensität wird die Extinktion $E$ der Probe berechnet.
Es gilt:
\begin{equation*}
    E = \log_{10} \left( \frac{I_0}{I} \right)
\end{equation*}

\subsection{Transmission, Absorption, und Extinktion}
Die Transmission $T$ ist definiert als das Verhältnis der Intensität des durch die Probe hindurchtretenden Lichts $I$ zur Intensität des einfallenden Lichts $I_0$.
Absorption $A$ ist der Prozess, bei dem Lichtenergie von Molekülen in der Probe aufgenommen wird.
Die Extinktion $E$ ist ein Maß für die Abschwächung des Lichts durch die Probe und wird durch die Summe von Absorption und Streuung verursacht.

Messtechnisch sind Absorption und Extinktion generell gleichwertig, da beide Größen die Abschwächung des Lichts beschreiben. sofern die Streuung vernachlässigbar ist.

\subsection{Gültigkeitsbereich des Lambert-Beerschen Gesetzes}
Das Lambert-Beersche Gesetz ist gültig unter bestimmten Bedingungen:
\begin{itemize}
    \item Die Lösung muss homogen und klar sein, ohne Partikel, die das Licht streuen könnten.
    \item Die Konzentration der absorbierenden Spezies sollte nicht zu hoch sein, da bei hohen Konzentrationen Abweichungen auftreten können.
    \item Das Licht sollte monochromatisch sein, also nur eine Wellenlänge besitzen.
    \item Die Küvette sollte eine konstante Schichtdicke haben und aus einem Material bestehen, das im relevanten Wellenlängenbereich transparent ist.
\end{itemize}

\newpage
\subsection{Isosbestische Punkte}
Ein isosbestischer Punkt ist eine Wellenlänge, bei der die Absorptionsspektren verschiedener Spezies oder Zustände einer Substanz sich schneiden.
Die Existenz eines isosbestischen Punktes deutet darauf hin, dass nur zwei Spezies oder Zustände in einem Gleichgewicht stehen.
In diesem Fall bedeutet das, dass Malachitgrün nur zu einer einzigen Carbinolbase reagiert, ohne dass weitere Zwischenprodukte oder Nebenreaktionen auftreten.
Die Anwesenheit eines isosbestischen Punktes lässt sich durch die Überlagerung der Absorptionsspektren der beiden Spezies erklären.

\newpage
\printbibliography
\newpage
\appendix

\section{Fehlerrechnung}
\subsection{Geschwindigkeitskonstanten}
Die Unsicherheit der Geschwindigkeitskonstanten $k$ wird mittels Gaußscher Fehlerfortpflanzung berechnet.
Hierbei wurde durch Betrachtung der Rohdaten der Zeitfehler auf \qty{0.05}{\second} geschätzt.
\begin{align*}
    k &= - \frac{\ln \frac{E}{E_0}}{t} \\
    \Rightarrow \Delta k &= \sqrt{ \left( \pdv{k}{t} \Delta t \right)^2 } \\
    \pdv{k}{t} &= \frac{\ln \frac{E}{E_0}}{t} \\
    \Delta k &= \sqrt{ \left( \frac{\ln \frac{E}{E_0}}{t} \Delta t \right)^2 } \\
    \intertext{Beispielhaft für pH = 10.513 (10.5):}
    \Delta k_{10.5} &= \sqrt{ \left( \frac{\ln \frac{0.5716}{0.6264}}{\qty{300}{\second}} \cdot \qty{0.05}{\second} \right)^2 } \\
    &\approx \qty{2.5e-5}{\per\second}
\end{align*}
Da die Unsicherheiten der einzelnen $k$ ungefähr um den gleichen Wert schwanken, wird dieser für alle $k$ als \qty{2.5e-5}{\per\second} angenommen.

\subsubsection{Geschwindigkeitskonstanten $k_1$ und $k_2$}
Die Unsicherheiten der Geschwindigkeitskonstanten $k_1$ und $k_2$ werden mittels linearer Regression auf Millimeterpapier bestimmt.
Hierbei ist die Unsicherheit der Steigung (für $k_1$) beziehungsweise des y-Achsenabschnitts (für $k_2$) gegeben durch die halbe Spanne der Extremgeraden (CI = 68\%).
\begin{align*}
    \Delta k_1 &= \frac{m_{\text{max}} - m_{\text{min}}}{2} \\
    &= \frac{\qty{0.867}{\liter\per\mol\per\second} - \qty{0.530}{\liter\per\mol\per\second}}{2} \\
    &= \qty{0.169}{\liter\per\mol\per\second} \\
    \Delta k_2 &= \frac{b_{\text{max}} - b_{\text{min}}}{2} \\
    &= \left| \frac{\qty{1.4e-4}{\per\second} - \qty{2.675e-4}{\per\second}}{2} \right| \\
    &= \qty{6.375e-5}{\per\second}
\end{align*}

\section{Kinetische Messungen}
An dieser Stelle sind die für die Bestimmung der Geschwindigkeitskonstanten verwendeten Plots dargestellt.
Alle Extinktionsmessungen wurden bei \qty{617}{\nano\meter} durchgeführt.
Zur besseren Veranschaulichung des zeitlichen Verlaufs wurde das Signal zusätzlich mittels nachfolgender Funktion geglättet.
\begin{minted}{python}
def _denoise(absorbances: List[float], window_size: int = 11) -> List[float]:
    """Simple moving average denoising."""
    abs_array = np.asarray(absorbances)
    kernel = np.ones(window_size) / window_size
    denoised = np.convolve(abs_array, kernel, mode='same')
    # remove first and last few points which are less reliable
    denoised[:window_size//2] = abs_array[:window_size//2]
    denoised[-(window_size//2):] = abs_array[-(window_size//2):]
    return denoised.tolist()
\end{minted}

\begin{figure}[H]
    \centering
    \includegraphics[width=\textwidth]{K2/plots/pH10_0.png}
    \caption{
        Aufnahme der Extinktion bei pH = 10.0 für $t = \qty{5}{\minute}$.
    }
\end{figure}

\begin{figure}[H]
    \centering
    \includegraphics[width=\textwidth]{K2/plots/pH10_3.png}
    \caption{
        Aufnahme der Extinktion bei pH = 10.3 für $t = \qty{5}{\minute}$.
    }
\end{figure}

\begin{figure}[H]
    \centering
    \includegraphics[width=\textwidth]{K2/plots/pH10_6.png}
    \caption{
        Aufnahme der Extinktion bei pH = 10.6 für $t = \qty{5}{\minute}$.
    }
\end{figure}

\begin{figure}[H]
    \centering
    \includegraphics[width=\textwidth]{K2/plots/pH11_03.png}
    \caption{
        Aufnahme der Extinktion bei pH = 11.0 für $t = \qty{5}{\minute}$.
    }
\end{figure}

\begin{figure}[H]
    \centering
    \includegraphics[width=\textwidth]{K2/plots/pH11_3.png}
    \caption{
        Aufnahme der Extinktion bei pH = 11.3 für $t = \qty{5}{\minute}$.
    }
\end{figure}

\includepdf[pages=-,landscape=true]{K2/plots/geschwindigkeitskonstante.pdf}
