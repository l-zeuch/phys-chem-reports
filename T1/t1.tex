\section{Theorie}
Ein Kalorimeter umfasst ein eingeschlossenes Volumen in einem Druckbehälter und ein Wasserbad, beide stehen im thermischen Kontakt zueinander.
Wird nun innerhalb des Volumens eine Probe verbrannt, so kann ein Temperaturanstieg im Wasserbad beobachtet werden.
Anhand dieses Anstieges können auf Basis von physikalisch-chemischen Zusammenhängen stoffspezifische Kenngrößen berechnet werden.
Damit die Probe möglichst vollständig verbrennt, wird das Volumen mit Sauerstoff im Überschuss befüllt.

\subsection{Wärmekapazität $c_v$ des Kalorimeters}
Zur genauen Berechnung ist es unbedingt notwendig, die Wärmekapazität des Kalorimeters zu kennen, da anhand der Temperaturdifferenz die Reaktionsenthalpie berechnet wird.
Eine kalorimetrische Messung wird generell adiabatisch durchgeführt.
$\Delta_R U$ sei bekannt, sodass
\begin{equation}
    c_v = - \frac{\Delta_R U}{\Delta T}\, \text{.} \label{eqn:heat-capacity}
\end{equation}

\subsection{Enthalpiebestimmungen}
Nach Kalibrierung des Kalorimeters kann nun die eigentliche Messung stattfinden.
Das unterliegende Prinzip ist ein Temperaturanstieg $\Delta T$, anhand dessen die relevanten Größen berechnet werden können.
\begin{align}
      \Delta_R U &= c_v \cdot \Delta T \label{eqn:verbrennungsenergie} \\
    \Delta_R U_m &= \frac{\Delta_R U}{n} \label{eqn:molare-verbrennungsenergie} \\
    \Delta_R H_m &= \Delta_R U_m + R \cdot T \cdot \Delta \nu \label{eqn:molare-verbrennungsenthalpie} \\
    \Delta_B H_m &= \nu_{\ch{CO2}} \cdot \Delta_B H_m (\ch{CO2}) + \nu_{\ch{H2O}} \cdot \Delta_B H_m (\ch{H2O}) - \Delta_R H_m \label{eqn:bildungsenthalpie}
\end{align}

\section{Sicherheitshinweise}
Die Kalorimeterbombe steht unter sehr hohem Druck.
Vor dem Öffnen der Bombe ist daher unbedingt sicherzustellen, dass der Druck vollständig über das Ventil abgelassen wurde.

\adjustbox{raise=2em}{\mbox{\stackengine{\Sstackgap}{%
\textbf{\Large Benzoesäure}}{\danger}{U}{l}{F}{T}{S}}}\hfill
\ghspic{acid}\ghspic{health} \\
\ghs*{h}{315}
\ghs*{h}{318}
\ghs*{h}{372}
\ghs*{p}{260}
\ghs*[dots=Haut]{p}{264}
\ghs*{p}{280}
\ghs*{p}{302+352}
\ghs*{p}{305+351+338}
\ghs*{p}{314}

\section{Durchführung}
\subsection{Kalibrierung des Kalorimeters}
Ein Baumwoll-Zündfaden wurde um den Heizdraht des Kalorimeters gewickelt, sodass die Enden des Fadens auf dem Boden des Probengefäßes liegen.
Ein Benzoesäure-Pressling von \qty{0.4972}{\gram} wurde auf die Probenaufnahme der Kalorimeterbombe gesetzt, sodass die Probe auf dem Zündfaden lag.
In den Bombenzylinder wurden \qty{5}{\milli\liter} Wasser eingefüllt.
Die Probenaufnahme wurde wieder eingesetzt und die Bombe handfest verschlossen.

Die Bombe wurde mit \qty{30}{\bar} Sauerstoff befüllt und in das Kalorimeter zurückgestellt.
Das Kalorimeter wurde verschlossen.
Die Temperatur des Wasserbades wurde aufgezeichnet, bis sich ein nahezu linearer Verlauf einstellte.
Am Kalorimeter wurde der Zündvorgang gestartet.
Die Temperatur wurde weiter aufgezeichnet, bis der Temperaturverlauf wieder annähernd linear war.

\subsection{Messung eines Reinstoffes und einer Lebensmittelprobe}
Es wurden \qty{0.5028}{\gram} Bernsteinsäure zwecks Herstellung eines Presslings eingewogen.
Die Masse des Presslings wurde anschließend bestimmt zu \qty{0.4843}{\gram}.
Die Probe wurde wie oben beschrieben vermessen.

Eine Lebensmittelprobe (\textit{TicTac}, $m =$ \qty{0.4665}{\gram}) wurde eingewogen und wie oben beschrieben vermessen.
Hierbei wurden die verwendeten Daten von Gruppe F bereitgestellt.

\section{Ergebnisse und Diskussion}

\subsection{Kalibrierung}

Die Wärmekapazität des Kalorimeters berechnet sich nach \autoref{eqn:heat-capacity} mit $\Delta_R U = $ \qty{- 26.439}{\kilo\joule\per\gram} für Benzoesäure\autocite{skript_t1} und der zusätzlichen Wärme des Zündfadens von \qty{50}{\joule} zu
\begin{align*}
    c_v &= - \frac{\Delta_R U}{\Delta T} \\ 
    c_v &= - \frac{\qty{- 26.439}{\kilo\joule\per\gram}\cdot \qty{0.4972}{\gram} + \qty{50}{\joule}}{\qty{1.29}{\kelvin}} \\
    c_v &= \qty{10.151 \pm 0.042}{\kilo\joule\per\kelvin}\, \text{.}
\end{align*}

\subsection{Chemischer Reinstoff (Bernsteinsäure)}
\subsubsection{Verbrennungsenergie}
Mit \autoref{eqn:verbrennungsenergie} berechnet sich die Verbrennungsenergie der verwendeten Bernsteinsäure zu
\begin{align*}
    \Delta_R U &= c_v \cdot \Delta T \\
               &= \qty{10.151}{\kilo\joule\per\kelvin} \cdot \qty{0.29}{\kelvin} \\
               &= \qty{2.944}{\kilo\joule}
\end{align*}

\subsubsection{Molare Verbrennungsenergie}
Es wurden \qty{0.4843}{\gram} Bernsteinsäure verbrannt, das entspricht \qty{0.004101}{\mole}.
\begin{align*}    
    \Delta_R U_m &= \frac{\Delta_R U}{n} \label{eqn:molare-verbrennungsenergie} \\
                 &= \frac{\qty{2.944}{\kilo\joule} + \qty{50}{\joule}}{\qty{0.004101}{\mole}} \\
                 &= \qty{2.93 \pm 0.07}{\kilo\joule}
\end{align*}

\subsubsection{Molare Verbrennungsenthalpie}
Bernsteinsäure verbrennt in der Sauerstoffatmosphäre der Kalorimeterbombe nach \autoref{scm:verbrennung-bernsteinsäure} vollständig zu Kohlenstoffdioxid und Wasser.
Es folgt für die Differenz der stöchiometrischen Faktoren $\Delta \nu$ der Gasteilchen $\Delta \nu = \frac{8+6-9}{2} = 2.5$.
\begin{scheme}[H]
    \centering
    \schemestart
    \setchemfig{atom sep=1.5em}
    2 \chemfig[angle increment=30]{HO-[-1](=[:270]O)-[1]-[-1]-[1](=[:90]O)-[-1]OH} + 9 \ch{O2} \ch{ -> } 8 \ch{CO2} + 6 \ch{H2O}
    \schemestop
    \caption{Verbrennung von Bernsteinsäure (\ch{C4H6O4}).}
    \label{scm:verbrennung-bernsteinsäure}
\end{scheme}

Es folgt somit mit \autoref{eqn:molare-verbrennungsenthalpie} für die molare Verbrennungsenthalpie
\begin{align*}
    \Delta_R H_m &= \Delta_R U_m + R \cdot T \cdot \Delta \nu \\ 
                 &= \qty{730.065}{\kilo\joule\per\mole} + R \cdot \qty{298.15}{\kelvin} \cdot 2.5 \\
                 &= \qty{736.262}{\kilo\joule\per\mole}
\end{align*}

\subsubsection{Molare Bildungsenthalpie}
Mittels den Bildungsenthalpien von Wasser (\qty{-285.98}{\kilo\joule\per\mole}) und Kohlenstoffdioxid (\qty{-393.42}{\kilo\joule\per\mole}) lässt sich die molare Bildungsenthalpie der Probe
über den Satz von Hess nach \autoref{eqn:bildungsenthalpie} berechnen.
\begin{align*}
    \Delta_B H_m &= \nu_{\ch{CO2}} \cdot \Delta_B H_m (\ch{CO2}) + \nu_{\ch{H2O}} \cdot \Delta_B H_m (\ch{H2O}) - \Delta_R H_m \\
                 &= (4 \cdot \qty{-393.42}{\kilo\joule\per\mole} + 3 \cdot \qty{- 285.98}{\kilo\joule\per\mole} - \qty{736.262}{\kilo\joule\per\mole}) \div 2 \\
                 &= \qty{-847.7 \pm 18.1}{\kilo\joule\per\mole}
\end{align*}
Die experimentell bestimmte molare Bildungsenthalpie stimmt im Rahmen des Fehlers mit dem Literaturwert \qty{823}{\kilo\joule\per\mole}\autocite{crc:std-succinic-acid} überein.
Abweichungen ergeben sich unter anderem aus dem Ablesefehler der grafischen Auswertung auf Millimeterpapier, da hier der relative Fehler recht groß ist und diese Größe weit am Anfang der Berechnungen einging.
Somit pflanzt sich der Fehler immer weiter fort und wird verstärkt.

\subsection{Lebensmittelprobe}
\subsubsection{Verbrennungsenergie}
Mit \autoref{eqn:verbrennungsenergie} berechnet sich die Verbrennungsenergie der Lebensmittelprobe zu
\begin{align*}
    \Delta_R U &= c_v \cdot \Delta T \\
               &= \qty{10.151}{\kilo\joule\per\kelvin} \cdot \qty{1.21}{\kelvin} \\
               &= \qty{12.282}{\kilo\joule}
\end{align*}
\subsubsection{Brennwert}
Die Masse der Lebensmittelprobe war $m = \qty{0.4665}{\gram}$.
Mit der Verbrennungsenergie und der Masse lässt sich so der Brennwert in $\frac{\unit{\kilo\joule}}{\qty{100}{\gram}}$ und $\frac{\unit{\kilo\cal}}{\qty{100}{\gram}}$ berechnen.
\begin{align*}
    H_S &= \frac{\qty{12.282}{\kilo\joule}}{\qty{0.4665}{\gram}} \\
        &= \qty{26.33}{\kilo\joule\per\gram} \\
        &= (\num{2.633 \pm 0.22}) \frac{\unit{\kilo\joule}}{\qty{100}{\gram}} \\ 
        &= (\num{0.629 \pm 0.001}) \frac{\unit{\kilo\cal}}{\qty{100}{\gram}} 
\end{align*}

\section{Zusammenfassung}
Das Kalorimeter wurde mittels Verbrennung einer bekannten Substanz, hier Benzoesäure, kalibriert.
Diese Kalibrierung wurde verwendet, um chemisch-physikalische Kenngrößen von Bernsteinsäure zu bestimmen.
Die berechnete molare Bildungsenthalpie der Bernsteinsäure wurde zu \qty{-847.7 \pm 18.1}{\kilo\joule\per\mole} bestimmt, mit Literaturwerten verglichen und bewertet.
Eine Lebensmittelprobe wurde ebenfalls kalorimetrisch vermessen und dessen Brennwert wurde zu $(\num{2.633 \pm 0.22}) \frac{\unit{\kilo\joule}}{\qty{100}{\gram}}$ bestimmt.

\newpage
\printbibliography
\newpage
\appendix

\section{Fehlerberechnung}
\subsection{Kalibrierung}

$c_v$ ist abhängig von $\Delta_R U$, $m$, und $\Delta T$.
$\Delta_R U$ ist ein Literaturwert und sei fehlerfrei.
$m$ wurde auf einer Analysenwaage mit vier Nachkommastellen bestimmt, der Fehler ist hier in der letzen Stelle, also \qty{\pm 0.0001}{\gram}.
$\Delta T$ wurde grafisch bestimmt; das Kalorimeter zeigte zwar vier Nachkommastellen an, die grafische Auswertung stellt allerdings nur zwei dar.
Hinzu kommt der Ablesefehler auf dem Millimeterpapier, was hier einen Fehler von \qty{\pm 0.005}{\kelvin} bedeutet.

Fehlerrechnung nach der Rechenregel für multiplikative Größen.\autocite{skript-physika}
\begin{align*}
    \frac{u_{c_v}}{c_v} &= \pm \left( \frac{u_m}{m} + \frac{u_{\Delta T}}{\Delta T} \right) \\
                      &= \pm \left( \frac{0.0001}{0.4972} + \frac{0.005}{1.29} \right) \\
                      &= \pm 4.077 \cdot 10^{-3} \Rightarrow \pm 0.4\% \\
                      &\implies \qty{10.151}{\kilo\joule\per\kelvin} \cdot 0.4\% = \qty{\pm 0.042}{\kilo\joule\per\kelvin} \\
                      &= \qty{10.151 \pm 0.042}{\kilo\joule\per\kelvin}
\end{align*}

\subsection{Reinstoff}
\subsubsection{Verbrennungsenergie}
Wie oben.
\begin{align*}
    \frac{u_{\Delta_R U}}{\Delta_R U} &= \pm \left( \frac{u_{c_v}}{c_v} + \frac{u_{\Delta T}}{\Delta T} \right) \\
                                      &= \pm \left( 4.077 \cdot 10^{-3} + \frac{0.005}{0.29} \right) \\
                                      &= \pm 0.0213 \Rightarrow \pm 2.13\% \\
                                      &\implies \qty{2.933}{\kilo\joule} \cdot 2.2\% = \qty{\pm 0.0616}{\kilo\joule} \\
                                      &= \qty{2.93 \pm 0.07}{\kilo\joule}
\end{align*}
\subsubsection{Molare Verbrennungsenergie}
Der Fehler der Stoffmenge fortgepflanzt über den Massenfehler wird vernachlässigbar.
\begin{align*}
    \frac{u_{\Delta_R U_m}}{\Delta_R U_m} &= \pm \left( \frac{u_{\Delta_R U}}{\Delta_R U} \right) \\
                                          &= \pm 0.0213 \\
                                          &\implies \qty{730.065}{\kilo\joule\per\mole} \cdot 2.2\% = \qty{15.33}{\kilo\joule\per\mole} \\
                                          &= \qty{730.1 \pm 15.4}{\kilo\joule\per\mole}
\end{align*}
\subsubsection{Molare Verbrennungsenthalpie}
Der Fehler der molaren Verbrennungsenthalpie wird primär gegeben durch den Fehler der molaren Verbrennungsenergie.
Da die Verbrennungsenthalpie bei $T = \qty{25}{\degreeCelsius}$ berechnet wird, ist diese Abhängigkeit fehlerfrei.
\begin{align*}
    \frac{u_{\Delta_R H_m}}{\Delta_R H_m} &= \pm 0.0213 \\
                                          &\implies \qty{736.262}{\kilo\joule\per\mole} \cdot 2.2\% = \qty{15.68}{\kilo\joule\per\mole} \\
                                          &= \qty{736.3 \pm 15.7}{\kilo\joule\per\mole}
\end{align*}

\subsubsection{Molare Bildungsenthalpie}
Da es sich um die Bildungsenthalpien für Wasser und Kohlenstoffdioxid um Literaturwerte handelt, seien diese als fehlerfrei behandelt.
Somit ist der Fehler der molaren Bildungsenthalpie ausschließlich abhängig vom Fehler der molaren Verbrennungsenthalpie.
\begin{align*}
    \frac{u_{\Delta_B H_m}}{\Delta_B H_m} &= \pm 0.0213 \\
                                          &\implies \qty{-847.679}{\kilo\joule\per\mole} \cdot 2.2\% = \qty{18.071}{\kilo\joule\per\mole} \\
                                          &= \qty{-847.7 \pm 18.1}{\kilo\joule\per\mole}
\end{align*}

\subsection{Lebensmittelprobe}
\begin{align*}
    \frac{u_{\Delta_R U}}{\Delta_R U} &= \pm \left( 4.077 \cdot 10^{-3} + \frac{0.005}{1.21} \right) \\
                                      &= \pm 8.209 \cdot 10^{-3} \\
                                      &\implies \qty{12.282}{\kilo\joule} \cdot 8.3\% = \qty{0.1}{\kilo\joule} \\
                                      &= \qty{12.3 \pm 0.1}{\kilo\joule} \\
    \frac{u_{H_S}}{H_S} &= \pm \left( 8.209 \cdot 10^{-3} + \frac{0.0001}{0.4665} \right) \\
                        &= \pm 8.209 \cdot 10^{-3} \\
                        &\implies 2.633 \frac{\unit{\kilo\joule}}{\qty{100}{\gram}} \cdot 8.3\% = 0.216 \frac{\unit{\kilo\joule}}{\qty{100}{\gram}} \\
                        &= (\num{2.633 \pm 0.22}) \frac{\unit{\kilo\joule}}{\qty{100}{\gram}} \\
                \intertext{beziehungsweise}
                        &= (\num{0.629 \pm 0.001}) \frac{\unit{\kilo\cal}}{\qty{100}{\gram}} 
\end{align*}
                           
\includepdf[landscape=true]{T1/Benzoesäure.pdf}
\includepdf[landscape=true]{T1/Bernsteinsäure.pdf}
\includepdf[landscape=true]{T1/Lebensmittelprobe.pdf}